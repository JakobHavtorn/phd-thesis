%!TEX root = ../thesis.tex
% LTeX: language=da-DK

\chapter[resumé (abstract in danish)]{Resumé (Abstract in Danish)}

Naturligt sprog spiller en nøglerolle i sundhedssystemer verden over. Alligevel har den medicinske interviewproces kun oplevet en lille udvikling sammenlignet med fremskridtene inden for medicinsk teknologi.
Traditionelt udføres medicinske interviews af sundhedspersonale, der primært afhænger af deres individuelle erfaring for at forstå en situation.
Med aldrende befolkninger, strenge dokumentationskrav, og fremskridt inden for diagnostiske muligheder og behandlingsmuligheder, er denne tilgang dyr og risikerer at komme til kort, hvilket potentielt kompromitterer nøjagtigheden og kvaliteten af medicinsk behandling.

Nylige fremskridt inden for naturlig sprogbehandling gør det muligt for maskinlæringsmodeller at deltage aktivt i medicinske interviews for at lette administrative byrder, forbedre dokumentation, og assistere i realtid.
Sundhedsplejen adskiller sig dog fra andre domæner på grund af den høje risiko forbundet med selv små fejl. Da ingen model er fejlfri, tilskynder dette til at associere modelprædiktioner med robuste usikkerhedsestimater, især i scenarier, der involverer ude-af-fordeling (UAF) data, såsom sjældne sygdomme.

Denne afhandling tager udgangspunkt i hovedhypotesen, at usuperviseret repræsentationslæring er nyttig til usikkerhedsestimering i medicinske opgaver.
Den giver følgende bidrag:
%
\begin{enumerate}[topsep=3pt, partopsep=0pt, itemsep=3pt, parsep=0pt, leftmargin=2em, label=(\alph*)] %, label=(\roman*)]
    \item En likelihood-ratio score til UAF-detektion med variationelle autoenkodere, der afhjælper den bevist negative effekt af lav-niveau features.
    \item En statistisk test til UAF-detektion, der kombinerer score- og typikalitetstests og kan bruges med likelihoods fra enhver differentiabel generativ model.
    \item Et benchmark for probabilistiske talerepræsentationsmodeller og en ny metode til at lære hierarkiske repræsentationer.
    \item En oversigt over usuperviseret repræsentationslæring til neural talebehandling og en tilsvarende modeltaksonomi.
    \item En fejlanalyse og revideret evaluering af state-of-the-art modeller til automatiseret medicinsk kodning på MIMIC-III og IV datasættene.
    \item Et retrospektivt studie af talebaseret genkendelse af stroke i præhospitale akuttelefonopkald der viser betydelig forbedring i forhold til opkaldstagere.
\end{enumerate}
%

Sammenfattende adresserer denne afhandling udfordringer indenfor usikkerhedsestimering og repræsentationslæring til tale og udforsker medicinske anvendelser af maskinlæring. 
Dens bidrag er afgørende i udviklingen af et operationelt beslutningsstøttesystem til medicinske interviews, der søger at øge kvaliteten af patientbehandlingen ved at understøtte effektiv, informeret beslutningstagen.
