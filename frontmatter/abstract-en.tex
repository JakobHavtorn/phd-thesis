%!TEX root = ../thesis.tex

\chapter[abstract]{Abstract}

Natural language plays a key role in healthcare systems worldwide; yet, the medical interview process has seen little development compared to the strides made in medical technology. 
Traditionally, medical interviews are conducted by healthcare professionals who primarily rely on their individual experience to understand a situation. 
Faced with aging populations, rigorous documentation requirements, and advances in diagnostic capabilities and treatment options, this approach is costly and risks falling short, potentially compromising the accuracy and quality of medical care.

Recent advances in natural language processing enable machine learning models to actively engage in medical interviews to alleviate administrative burdens, enhance documentation, and provide real-time assistance in dialogue. 
However, healthcare stands apart from other domains due to the high risk associated with even minor errors. Since no model is error-free, this impels pairing model predictions with robust uncertainty estimates, especially in scenarios involving out-of-distribution (OOD) data, such as rare illnesses. 

This thesis sets out from the main hypothesis that unsupervised representation learning is useful for uncertainty estimation for medical tasks. 
It makes the following contributions:
%
\begin{enumerate}[topsep=3pt, partopsep=0pt, itemsep=3pt, parsep=0pt, leftmargin=2em, label=(\alph*)] %, label=(\roman*)]
    \item A likelihood-ratio score for OOD detection with variational autoencoders that alleviates the proved impeding effect of low-level features.
    \item A statistical test for OOD detection combining the score and typicality tests and is applicable with likelihoods from any differentiable generative model.
    \item A benchmark of probabilistic speech representation learners and a novel method to learn hierarchical representations.
    \item An overview of unsupervised representation learning for neural speech processing and a corresponding model taxonomy.
    \item An error analysis and revised evaluation of state-of-the-art models for automated medical coding on the MIMIC-III and IV datasets. 
    \item A retrospective study of speech-based stroke recognition in prehospital medical helpline calls with significant improvements over call-taker performance.
\end{enumerate}
%
In summary, this thesis addresses challenges in uncertainty estimation and representation learning for speech while exploring medical applications of machine learning.
Its contributions are vital in the development of an operational decision support system for medical interviews, ultimately aiming to improve the quality of patient care by supporting effective, informed decision-making.
