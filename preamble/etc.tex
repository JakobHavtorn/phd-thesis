%!TEX root = ../thesis.tex

% Content specific packages.

\usepackage{blindtext}
\usepackage{algorithm}
\usepackage{algpseudocode}
\usepackage{multicol}
\usepackage{multirow}
\usepackage{makecell}
\usepackage{booktabs}
\usepackage{enumerate}
\usepackage{enumitem}
% \usepackage{layouts}
\usepackage[showframe=\showframe,xetex]{geometry}
\usepackage{adjustbox}
\usepackage{siunitx}
\usepackage{nomencl}
\usepackage{amsmath}
\usepackage{amssymb}
\usepackage[acronym,toc]{glossaries}
%!TEX root = ../thesis.tex

% \makeglossaries
\makenoidxglossaries


% Glossary

\newglossaryentry{encoder}
{
    name=encoder,
    description={A general term for a function that maps an input to a different representation.}
}

\newglossaryentry{decoder}
{
    name=decoder,
    description={A general term for a function that maps a representation to an output.}
}


% Abbreviations
\newacronym{AI}{AI}{artificial intelligence}

\newacronym{MLP}{MLP}{multilayer perceptron}
\newacronym{FNN}{FNN}{feedforward neural network}
\newacronym{CNN}{CNN}{convolutional neural network}
\newacronym{RNN}{RNN}{recurrent neural network}
\newacronym{LSTM}{LSTM}{long short-term memory}
\newacronym{GRU}{GRU}{gated recurrent unit}
\newacronym{BNN}{BNN}{Bayesian neural network}

\newacronym{ReLU}{ReLU}{rectified linear unit}

\newacronym{SGD}{SGD}{stochastic gradient descent}

\newacronym{SGVB}{SGVB}{stochastic gradient variational Bayes}
\newacronym{VAE}{VAE}{variational autoencoder}
\newacronym{LVM}{LVM}{latent variable model}
\newacronym{KL}{KL}{Kullback-Leibler}

\newacronym{NLL}{NLL}{negative log-likelihood}
\newacronym{MSE}{MSE}{mean squared error}
\newacronym{CCE}{CCE}{categorical cross entropy}
\newacronym{MLE}{MLE}{maximum likelihood estimate}

\newacronym{PCA}{PCA}{principal components analysis}

\newacronym{PDF}{PDF}{probability density function}
\newacronym{PMF}{PMF}{probability mass function}

\newacronym{iid}{iid}{independent and identically distributed}

\newacronym{CPU}{CPU}{central processing unit}
\newacronym{GPU}{GPU}{graphics processing unit}
\newacronym{HPC}{HPC}{high performance computing}



%\usepackage{pgfplots}                 % Plot tools
%\pgfplotsset{compat=1.7}
\usetikzlibrary{
    arrows,
    matrix,
    positioning,
    shapes,
    shapes.multipart,
    topaths,
    bayesnet,
}
% tikzlibrary.code.tex
%
% Copyright 2010-2011 by Laura Dietz
% Copyright 2012 by Jaakko Luttinen
%
% This file may be distributed and/or modified
%
% 1. under the LaTeX Project Public License and/or
% 2. under the GNU General Public License.
%
% See the files LICENSE_LPPL and LICENSE_GPL for more details.

% Load other libraries
\usetikzlibrary{shapes}
\usetikzlibrary{fit}
\usetikzlibrary{chains}
\usetikzlibrary{arrows}

% Latent node
\tikzstyle{latent} = [circle,fill=white,draw=black,inner sep=1pt,
minimum size=20pt, font=\fontsize{10}{10}\selectfont, node distance=1]
% Observed node
\tikzstyle{obs} = [latent,fill=gray!25]
% Constant node
\tikzstyle{const} = [rectangle, inner sep=0pt, node distance=1]
% Factor node
\tikzstyle{factor} = [rectangle, fill=black,minimum size=5pt, inner
sep=0pt, node distance=0.4]
% Deterministic node
\tikzstyle{det} = [latent, diamond]

% Plate node
\tikzstyle{plate} = [draw, rectangle, rounded corners, fit=#1]
% Invisible wrapper node
\tikzstyle{wrap} = [inner sep=0pt, fit=#1]
% Gate
\tikzstyle{gate} = [draw, rectangle, dashed, fit=#1]

% Caption node
\tikzstyle{caption} = [font=\footnotesize, node distance=0] %
\tikzstyle{plate caption} = [caption, node distance=0, inner sep=0pt,
below left=5pt and 0pt of #1.south east] %
\tikzstyle{factor caption} = [caption] %
\tikzstyle{every label} += [caption] %

\tikzset{>={triangle 45}}

%\pgfdeclarelayer{b}
%\pgfdeclarelayer{f}
%\pgfsetlayers{b,main,f}

% \factoredge [options] {inputs} {factors} {outputs}
\renewcommand{\factoredge}[4][]{ %
  % Connect all nodes #2 to all nodes #4 via all factors #3.
  \foreach \f in {#3} { %
    \foreach \x in {#2} { %
      \path (\x) edge[-,#1] (\f) ; %
      %\draw[-,#1] (\x) edge[-] (\f) ; %
    } ;
    \foreach \y in {#4} { %
      \path (\f) edge[->,#1] (\y) ; %
      %\draw[->,#1] (\f) -- (\y) ; %
    } ;
  } ;
}

% \edge [options] {inputs} {outputs}
\renewcommand{\edge}[3][]{ %
  % Connect all nodes #2 to all nodes #3.
  \foreach \x in {#2} { %
    \foreach \y in {#3} { %
      \path (\x) edge [->,#1] (\y) ;%
      %\draw[->,#1] (\x) -- (\y) ;%
    } ;
  } ;
}

% \factor [options] {name} {caption} {inputs} {outputs}
\renewcommand{\factor}[5][]{ %
  % Draw the factor node. Use alias to allow empty names.
  \node[factor, label={[name=#2-caption]#3}, name=#2, #1,
  alias=#2-alias] {} ; %
  % Connect all inputs to outputs via this factor
  \factoredge {#4} {#2-alias} {#5} ; %
}

% \plate [options] {name} {fitlist} {caption}
\renewcommand{\plate}[4][]{ %
  \node[wrap=#3] (#2-wrap) {}; %
  \node[plate caption=#2-wrap] (#2-caption) {#4}; %
  \node[plate=(#2-wrap)(#2-caption), #1] (#2) {}; %
}

% \gate [options] {name} {fitlist} {inputs}
\renewcommand{\gate}[4][]{ %
  \node[gate=#3, name=#2, #1, alias=#2-alias] {}; %
  \foreach \x in {#4} { %
    \draw [-*,thick] (\x) -- (#2-alias); %
  } ;%
}

% \vgate {name} {fitlist-left} {caption-left} {fitlist-right}
% {caption-right} {inputs}
\renewcommand{\vgate}[6]{ %
  % Wrap the left and right parts
  \node[wrap=#2] (#1-left) {}; %
  \node[wrap=#4] (#1-right) {}; %
  % Draw the gate
  \node[gate=(#1-left)(#1-right)] (#1) {}; %
  % Add captions
  \node[caption, below left=of #1.north ] (#1-left-caption)
  {#3}; %
  \node[caption, below right=of #1.north ] (#1-right-caption)
  {#5}; %
  % Draw middle separation
  \draw [-, dashed] (#1.north) -- (#1.south); %
  % Draw inputs
  \foreach \x in {#6} { %
    \draw [-*,thick] (\x) -- (#1); %
  } ;%
}

% \hgate {name} {fitlist-top} {caption-top} {fitlist-bottom}
% {caption-bottom} {inputs}
\renewcommand{\hgate}[6]{ %
  % Wrap the left and right parts
  \node[wrap=#2] (#1-top) {}; %
  \node[wrap=#4] (#1-bottom) {}; %
  % Draw the gate
  \node[gate=(#1-top)(#1-bottom)] (#1) {}; %
  % Add captions
  \node[caption, above right=of #1.west ] (#1-top-caption)
  {#3}; %
  \node[caption, below right=of #1.west ] (#1-bottom-caption)
  {#5}; %
  % Draw middle separation
  \draw [-, dashed] (#1.west) -- (#1.east); %
  % Draw inputs
  \foreach \x in {#6} { %
    \draw [-*,thick] (\x) -- (#1); %
  } ;%
}



\definecolor{customgreen} {RGB}{217	234	212}
\definecolor{customblue}  {RGB}{205	226	242}
\definecolor{custommorered}{RGB}{218 46 42}
\definecolor{customred}{RGB}{255 182 173}%{255 149 150}

\colorlet{observed-color}{customgreen}
\colorlet{latent-color}{customblue}
\colorlet{deterministic-color}{gray!15}
\colorlet{deterministic-skip-color}{custommorered}
\colorlet{shared-function-color}{blue}


% Listings
\lstset{
    basicstyle=\footnotesize\ttfamily,% the size of the fonts that are used for the code
    breakatwhitespace=false,          % sets if automatic breaks should only happen at whitespace
    breaklines=true,                  % sets automatic line breaking
    captionpos=b,                     % sets the caption-position to bottom
    commentstyle=\color{s14a},        % comment style
    deletekeywords={},                % if you want to delete keywords from the given language
    escapeinside={\%*}{*)},           % if you want to add LaTeX within your code
    frame=single,                     % adds a frame around the code
    keywordstyle=\bfseries\ttfamily\color{s09}, % keyword style
    language=Python,                  % the language of the code
    morekeywords={*,...},             % if you want to add more keywords to the set
    numbers=left,                     % where to put the line-numbers; possible values are (none, left, right)
    numbersep=5pt,                    % how far the line-numbers are from the code
    numberstyle=\sffamily\tiny\color{dtugray}, % the style that is used for the line-numbers
    rulecolor=\color{dtugray},        % if not set, the frame-color may be changed on line-breaks within not-black text (e.g. comments (green here))
    showspaces=false,                 % show spaces everywhere adding particular underscores; it overrides 'showstringspaces'
    showstringspaces=false,           % underline spaces within strings only
    showtabs=false,                   % show tabs within strings adding particular underscores
    stepnumber=1,                     % the step between two line-numbers. If it's 1, each line will be numbered
    stringstyle=\color{s07},          % string literal style
    tabsize=2,                        % sets default tabsize to 2 spaces
    title=\lstname,                   % show the filename of files included with \lstinputlisting; also try caption instead of title
}

