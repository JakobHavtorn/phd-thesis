%!TEX root = ../thesis.tex

% Content specific packages.

\usepackage{blindtext}
\usepackage{algorithm}
\usepackage{algpseudocode}
\usepackage{multicol}
\usepackage{booktabs}
\usepackage{multirow}
\usepackage{enumerate}
\usepackage{enumitem}
% \usepackage{layouts}
\usepackage[showframe=\showframe,xetex]{geometry}

\usepackage[acronym,toc]{glossaries}
%!TEX root = ../thesis.tex

% \makeglossaries
\makenoidxglossaries


% Glossary

\newglossaryentry{encoder}
{
    name=encoder,
    description={A general term for a function that maps an input to a different representation.}
}

\newglossaryentry{decoder}
{
    name=decoder,
    description={A general term for a function that maps a representation to an output.}
}


% Abbreviations
\newacronym{AI}{AI}{artificial intelligence}

\newacronym{MLP}{MLP}{multilayer perceptron}
\newacronym{FNN}{FNN}{feedforward neural network}
\newacronym{CNN}{CNN}{convolutional neural network}
\newacronym{RNN}{RNN}{recurrent neural network}
\newacronym{LSTM}{LSTM}{long short-term memory}
\newacronym{GRU}{GRU}{gated recurrent unit}
\newacronym{BNN}{BNN}{Bayesian neural network}

\newacronym{ReLU}{ReLU}{rectified linear unit}

\newacronym{SGD}{SGD}{stochastic gradient descent}

\newacronym{SGVB}{SGVB}{stochastic gradient variational Bayes}
\newacronym{VAE}{VAE}{variational autoencoder}
\newacronym{LVM}{LVM}{latent variable model}
\newacronym{KL}{KL}{Kullback-Leibler}

\newacronym{NLL}{NLL}{negative log-likelihood}
\newacronym{MSE}{MSE}{mean squared error}
\newacronym{CCE}{CCE}{categorical cross entropy}
\newacronym{MLE}{MLE}{maximum likelihood estimate}

\newacronym{PCA}{PCA}{principal components analysis}

\newacronym{PDF}{PDF}{probability density function}
\newacronym{PMF}{PMF}{probability mass function}

\newacronym{iid}{iid}{independent and identically distributed}

\newacronym{CPU}{CPU}{central processing unit}
\newacronym{GPU}{GPU}{graphics processing unit}
\newacronym{HPC}{HPC}{high performance computing}




%\usepackage{pgfplots}                 % Plot tools
%\pgfplotsset{compat=1.7}
\usetikzlibrary{
    arrows,
    matrix,
    positioning,
    shapes,
    topaths,
    bayesnet,
}

% Listings
\lstset{
    basicstyle=\footnotesize\ttfamily,% the size of the fonts that are used for the code
    breakatwhitespace=false,          % sets if automatic breaks should only happen at whitespace
    breaklines=true,                  % sets automatic line breaking
    captionpos=b,                     % sets the caption-position to bottom
    commentstyle=\color{s14a},        % comment style
    deletekeywords={},                % if you want to delete keywords from the given language
    escapeinside={\%*}{*)},           % if you want to add LaTeX within your code
    frame=single,                     % adds a frame around the code
    keywordstyle=\bfseries\ttfamily\color{s09}, % keyword style
    language=Python,                  % the language of the code
    morekeywords={*,...},             % if you want to add more keywords to the set
    numbers=left,                     % where to put the line-numbers; possible values are (none, left, right)
    numbersep=5pt,                    % how far the line-numbers are from the code
    numberstyle=\sffamily\tiny\color{dtugray}, % the style that is used for the line-numbers
    rulecolor=\color{dtugray},        % if not set, the frame-color may be changed on line-breaks within not-black text (e.g. comments (green here))
    showspaces=false,                 % show spaces everywhere adding particular underscores; it overrides 'showstringspaces'
    showstringspaces=false,           % underline spaces within strings only
    showtabs=false,                   % show tabs within strings adding particular underscores
    stepnumber=1,                     % the step between two line-numbers. If it's 1, each line will be numbered
    stringstyle=\color{s07},          % string literal style
    tabsize=2,                        % sets default tabsize to 2 spaces
    title=\lstname,                   % show the filename of files included with \lstinputlisting; also try caption instead of title
}

