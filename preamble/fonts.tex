%!TEX root = ../thesis.tex

% Math symbols
% \usepackage{amsmath}
% \usepackage{amssymb}

% Text fonts (http://www.macfreek.nl/memory/Fonts_in_LaTeX)
% Install fonts from /usr/local/texlive/<version>/texmf-dist/fonts/opentype/public
% \usepackage[utf8]{inputenc}
\usepackage{fontspec}
% \usepackage[T1]{fontenc}
% \usepackage{lmodern}
\usepackage{slantsc}
% \RequirePackage{fix-cm}
\usepackage{bold-extra}
\usepackage{csquotes}       % language sensitive quotation facilities


% Method that works with TeXLive 2023
% \usepackage[math-style=upright]{unicode-math}
% \setmainfont{texgyrepagella-regular.otf}[Scale=1.0, Ligatures={Common,Rare,TeX}]  % Or Palatino Linotype, etc.
% \setmathfont{texgyrepagella-math.otf}[Scale=MatchLowercase]
% \setmathfont{euler.otf}[range={up/{Latin,latin,Greek,Greek}, bfup/{Latin,latin,Greek,Greek}, cal, bfcal, frak, bffrak}, Scale=MatchLowercase, script-features={}, sscript-features={}]


% Method that works with TeXLive 2023 and is simpler and has fewer missing characters
\usepackage{newpxtext, eulerpx}


% Method that should be best practice for TeXLive
% These packages were recommended but seem to cause trouble with the euler-math and unicode-math packages.
% \usepackage{newpxtext,newpxmath}
% \setmainfont{texgyrepagella-regular.otf}[Scale=1.0, Ligatures={Common,Rare,TeX}]  % Or Palatino Linotype, etc.
% \usepackage{euler-math}


% Method that works with TeXLive 2022 (but is missing some special characters e.g. ê, ç)
% \usepackage{amssymb}
% \usepackage{upgreek}
% \usepackage{mathpazo}
% \usepackage[OT1,euler-digits]{eulervm}
% \renewcommand{\mathbf}{\mathbold}  % euler requires \mathbold for bold math





% % Remove: "Font shape `T1/eulervm/m/n' undefined (Font) using `T1/cmr/m/n' instead."
% \usepackage{substitutefont}
% \substitutefont{TS1}{eulervm}{cmr}

%
% [
%   Extension=.otf,
%   UprightFont=*-regular,
%   ItalicFont=*-italic,
%   BoldFont=*-bold,
%   BoldItalicFont=*-bolditalic,
%   BoldSmallCapsFont=*-boldsmallcaps,
%   Numbers=OldStyle,
% ]

% \setmainfont{QTPalatine}

% \DeclareCharacterInheritance
%    { encoding = {TU,EU1,EU2},
%      family   = {QTPalatine} }
%    { A = {\`A,\'A,\^A,\~A,\"A,\r A},
%      a = {\`a,\'a,\^a,\~a,\"a,\r a},
%      C = {\c C},
%      c = {\c c},
%      D = {\DH},
%      d = {\dj},
%      E = {\`E,\'E,\^E,\"E},
%      e = {\`e,\'e,\^e,\"e},
%      I = {\`I,\'I,\^I,\"I},
%      i = {\`i,\'i,\^i,\"i,\i},
%      L = {\L},
%      l = {\l},
%      N = {\~N},
%      n = {\~n},
%      O = {\O,\`O,\'O,\^O,\~O,\"O},
%      o = {\o,\`o,\'o,\^o,\~o,\"o},
%      S = {\v S},
%      s = {\v s},
%      U = {\`U,\'U,\^U,\"U},
%      u = {\`u,\'u,\^u,\"u},
%      Y = {\'Y,\"Y},
%      y = {\'y,\"y},
%      Z = {\v Z},
%      z = {\v z}
%    }


% % Sans-serif font
% \setsansfont[
%     Ligatures=TeX,
%     Extension=.otf,
%     UprightFont=*-regular,
%     BoldFont=*-bold,
%     ItalicFont=*-italic,
%     BoldItalicFont=*-bolditalic,
%     % SlantedFont=,
%     % BoldSlantedFont=,
%     % SmallCapsFont=
%     Scale=0.8      % Adjustmens when using math in sections
% ]{texgyreadventor}


% Monospaced
% \setmonofont[Scale=MatchLowercase]{Linux Biolinum O}

%\setsansfont[Ligatures=TeX]{Neo Sans Intel}    % Neo Sans Intel – Like DTU font but more symbols
%\setsansfont[
%    Ligatures=TeX,
%    Scale=0.8
%]{NeoSans}           % NeoSans – DTU font (missing `+' symbols and other)
% \setsansfont[Ligatures=TeX]{CMU Sans Serif}    % Computer Modern Unicode font
%\setsansfont[Ligatures=TeX]{Latin Modern Sans} % Latin Modern Sans serif font

% Use this for more convienent sans serif font in math mode.
%\setmathsf{Latin Modern Sans}
