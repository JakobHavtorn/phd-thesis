%!TEX root = ../thesis.tex

\section{Analysis of Self-Supervised Representations}
\label{analysis}

% {\color{blue} Katrin}\\
The previous sections have shown how self-supervised learning can result in
powerful representations that provide a robust starting point for several
downstream tasks. It is natural to ask if we can gain an even deeper
understanding of the nature of these representations, in order to further
optimize them or apply them to different problems.
What is the information encoded in these representations? How robust are they
to distributional shifts, and how dependent are they on the size of the
training data? Do they generalize across languages? What are the key
ingredients for training powerful representations: input data, network
architecture, training criterion, or all three? Can we predict their
performance on downstream tasks from their training behavior? This section
tries to answer these questions by summarizing several studies that analyze
self-supervised representations.

\subsection{Information content}

In \cite{pasad_layerwise_2021} wav2vec~2.0 representations were analyzed with
respect to their acoustic-linguistic information content at different
network layers. Three different mechanisms were used for this
purpose. The first of these is canonical correlation analysis (CCA),
which computes similarity scores between two continuous vectors
based on the maximum correlation of their linear projections.  These
can be used to judge the similarity of embeddings at different layers
with each other, with standard acoustic representations such as mel
filterbank features, or word embeddings derived from text. The second
method clusters continuous representation vectors and computes the
discrete mutual information between cluster IDs and phone or word
labels. The third method involves probing tasks: representation
vectors extracted from the network are used to perform simple
downstream tasks, in particular determining whether two acoustic segments
correspond to the same word, and a standard benchmark of 11 word
similarity tasks~\cite{faruqi14}. These are mostly used to gauge the
amount of lexical information present in the embeddings.  Using this
battery of tests the authors compared pre-trained models of varying
sizes as well as models fine-tuned for ASR. They found that pre-trained
models show an autoencoder-style behavior, with early layers showing
strong similarity with input features, intermediate layers diverging
more, and final layers reverting back to higher similarity with input
features and early layers. Generally, the earlier layers in wav2vec~2.0 
models encode acoustic information. The next set of layers encodes
phonetic class information, followed by word meaning information,
before reverting back to encoding phonetic/acoustic information. Thus,
extracting representations from the last layers for tasks that require
phonetic or word-related information may not be the best
strategy. Indeed, the authors of \cite{baevski2021unsupervised} show that a
phone classifier trained on each of the 24 frozen layers of a wav2vec~2.0 model
showed the lowest phone error rates
for layers 10--21 and higher error rates for the other layers. 
\cite{pasad_layerwise_2021} further show that fine-tuning the pre-trained model with a
character-level CTC training criterion changes the behavior of the
last layers (especially the final two layers), breaking the
autoencoder-style behavior and focusing the information encoded in the
last layers on orthographic-phonetic and word information. 

The peaking of class-relevant information in intermediate layers seems to be
common across different self-supervised learners and different modalities. In
an analysis of text-based Transformers trained with a masked language model
criterion~\cite{voita2019} observed a similar compression plus reconstruction
pattern. Interestingly, similar network behavior was also recently described
for self-supervised learners in computer vision: using a contrastive
self-supervised learner (SimCLR) that optimizes for augmentation invariance,
\cite{grigg2021self} show that it is the intermediate representations that most
closely approximate information learned in a supervised way, i.e., they
provide more class information than the representations from final layers. This
is similar to the findings described above for wav2vec~2.0 without fine-tuning,
where intermediate layers provide more information about phone and word
classes.

Self-supervised representations may encode other information besides phonetic
classes or words, for example, channel, language, speaker, and sentiment
information. 
\edit{It is shown that the per-utterance mean of CPC features captures speaker information to a large extent\cite{van2021analyzing}.}
Location of information pertaining to speakers vs.\ language classes was
analyzed in \cite{ling20odyssey} for a 12-layer BERTphone model. This model
combines a self-supervised masked reconstruction loss with a phone-based CTC
loss to produce representations	for speaker recognition and language
identification. By	analyzing the weights	of a linear combination	of layer
representations for these two downstream tasks, it was	shown that language
recognition draws on representation	from higher layers (peaking at layer~10)
whereas speaker recognition benefited from layers at positions 6, 9, and 12.
This may indicate that language recognition relies more on higher-level
phonetic information whereas speaker recognition uses a combination of
acoustic and phonetic information. In a recent study~\cite{chen_wavlm_2021} the
same technique was used to identify layer contributions for the downstream
SUPERB benchmark tasks in the WavLM model. For a smaller model (95M parameters)
it was again confirmed that lower layers encode speaker-related information
necessary for speaker diarization and verification whereas higher layers encode
phonetic and semantic information. Another study~\cite{wang2021layersup} used
explicit self-supervised loss at the intermediate layers rather than just the
output layer of a HuBERT model in order to enforce better learning of phonetic
information. The resulting model was indeed better at downstream tasks
requiring information about phonetic content, such as phone recognition, ASR,
and keyword spotting, but worse at speaker-related tasks like speaker
diarization and verification. 

%Hung-yi Lee: My group also has a paper analyzing the attention of self-supervised speech model (https://arxiv.org/abs/2006.03265). I will include some findings in this subsection later.
Most self-supervised learning approaches rely on a Transformer architecture for
the representation model. In \cite{AnalyzeAttention} the attention patterns in
generatively trained Transformer representation models were analyzed.
Self-attention heads were grouped into three categories: diagonal, vertical,
and global. It was found that the diagonal head focuses on neighbors and is
highly correlated with phoneme boundaries, whereas the vertical head focuses on
specific phonemes in the utterance. Global heads were found to be redundant as
removing them resulted in faster inference time and higher performance.

\subsection{Training criterion}
In \cite{chung2021}, representations based on different training criteria
(masked predictive coding, contrastive predictive coding, and autoregressive
predictive coding) were compared  and analyzed with respect to the correlation
between their training loss and performance on both phone discrimination and
speaker classification probing tasks. It was observed that the autoregressive
predictive coding loss showed the strongest correlation with downstream
performance on both tasks; however, models were not further analyzed
internally. An evaluation of the similarity of representations trained
according to the three criteria above (but with different architectures and
directionality of contextual information) also showed that it is the training
criterion that most influences the information encoded in the representations,
not the architecture of the learner or the directionality of the input. 

A similar insight was obtained in \cite{zhou2020}, which compared vq-vae and
vq-wav2vec with respect to their ability to discover phonetic units.
The vq-vae model extracts continuous features from the audio signal; a
quantizer then  maps them into a discrete space, and a decoder is trained to
reconstruct the original audio conditioned on the latent discrete
representation and the past acoustic observations. By contrast, vq-wav2vec
predicts future latent discrete representations based on contextualized
embeddings of past discrete representations, in a CPC-style way. The models
were evaluated according to their ability to discover phonetic units (as
measured by phone recognition error rate on TIMIT, and the ZeroSpeech ABX task
(see \cref{sec:zero} for more details)), and it was found that the predictive vq-wav2vec
model fared better than the autoencoder-like vq-vae model, most likely due to
its superior ability to model temporal dynamics.

\subsection{Effects of data and model size}
\label{subsec:modelsize}

How does the performance of self-supervised models change in relation to the
amount of training data, and in relation to the size (number of parameters) of
the model?
Several studies have demonstrated better downstream performance when using
larger datasets~\cite{rivi20, kawakami_learning_2020, chen_wavlm_2021}. For
example,
\cite{kawakami_learning_2020} compared representations learned by a bidirectional
CPC model from the standard 960 hour LS corpus and a corpus of 8,000 hours of
diverse speech from multiple sources.\textsuperscript{\ref{footnote:CPC-8k}}
Not surprisingly, an ASR model trained on top of these representations
performed better when representations were learned from the larger dataset. 
Although the precise relationship between data size and performance has not
been quantified, we can assume that it follows a law of diminishing returns (or
power law), 
similar to observations for most data-intensive machine learning tasks.
In addition to the size of the dataset, the diversity of the data also seems
to play a role, although this was not quantified in this study. However, recent
experiments with larger and more diverse data collections~\cite{chen_wavlm_2021}
confirm this assumption, as do 
explicit investigations of domain shift robustness (see
\cref{subsec:robustness} below). 

\edit{The relation between model sizes and downstream performances have also been investigated \cite{pue2021scaling,versteegh2015zero}.}
Using the Mockingjay 
model~\cite{liu_mockingjay_2020}, the authors in \cite{pue2021scaling} attempt to establish a relationship between
model size and self-supervised \ensuremath{L_1} loss and demonstrate that it approximately
follows a power law. Model size and accuracy on downstream phone
classification and speaker recognition tasks are positively correlated but do not
exactly follow a power law; rather, the accuracy saturates as models increase in
size, possibly due to the lack of a corresponding expansion in training data
size.

\subsection{Robustness and transferability}
\label{subsec:robustness}

It is well known that traditional speech features like MFCCs lack
robustness against environmental effects such as additive noise,
reverberation, accents, etc., that cause differences in the distributions of
speech features.
Do pre-trained representations offer greater robustness against distributional
shifts? 
One study~\cite{kawakami_learning_2020} compared pre-trained
representations from a CPC model against MFCCs and 
found pre-trained representations to be more robust to mismatches between
training and test data. The 
training data consisted of clean, read speech (LS)
whereas test data consisted of the Switchboard corpus and TED talks. The
distributional shifts here may stem from both the acoustics (microphone, room
reverberation) as well as
lexical effects related to topic and style, as well as differences in speaker
characteristics such as accent. 
Similar problems were also investigated using HuBERT and wav2vec~2.0 models in
\cite{chang2021exploration}.
In \cite{robustw2v2} domain effects were studied in greater detail using 
datasets from six different domains. In particular, the authors focused on the
usefulness of adding out-of-domain data to pre-training. The general
conclusions are that pre-training on more and diverse domains is preferable:
models pre-trained on more domains performed better than those pre-trained on
fewer when tested on held-out domains, regardless of which additional
labeled data was used for fine-tuning. Adding in-domain unlabeled data---if
available---to pre-training improves performance robustly; however, even
out-of-domain unlabeled data is helpful and closes  66--73\% of the performance
gap between the ideal setting of in-domain labeled data and a competitive
supervised out-of-domain model. 
 
In \cite{rivi20} the effectiveness of CPC-trained representations for
phone discrimination tasks was compared across several languages. It
was found that representations pre-trained only on English successfully
enabled phone discrimination in 10 other languages, rivaling
supervised methods in accuracy in low-data regimes (1h of labeled data
per language). Thus, self-supervised pre-training enables the model to
learn contextualized speech features that generalize across different
languages. In \cite{conneau20xling}, a wav2vec~2.0 model was trained on data
from multiple different languages and different corpora (Babel, Common Voice,
and multilingual LS)
jointly, followed by fine-tuning for each individual language. The largest
model covers 53 languages in total
and consists of 56,000 hours of speech. Compared to monolingual pre-training,
even smaller models trained on only ten languages improve performance
substantially on a downstream character-based ASR task. Low-resource languages
with little labeled data improve the most under this training regime.
Multilingual representations also resulted in competitive performance (lower
character error rate than monolingual representations) for languages not
present in the training dataset, again showing that unsupervised pre-trained
representations can learn generic features of the speech signal that generalize
across different languages. The study also found that sharing data from closely
related languages is more beneficial than combining distant languages. An
analysis of language clusters in the shared discrete latent representation
space revealed that similar languages do indeed show a higher degree of sharing
of discrete tokens. 
Finally, one might ask whether the interpretation of representations extracted
from different layers of a self-supervised models also generalizes to the
multilingual setting. Experiments in \cite{baevski2021unsupervised} on phone
recognition in eight languages based on the different layers of the
multilingual wav2vec~2.0 XLSR-53 model indicate that this is indeed the case:
phone error rates showed the same pattern as in the monolingual (English)
scenario, with lower phone error rates for middle layers as opposed to
earlier/later layers. 





