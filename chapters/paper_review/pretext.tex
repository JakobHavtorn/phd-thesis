%!TEX root = ../thesis.tex



\begin{table}
    \centering
    \caption[An overview of approaches within the three main categories of
    self"=supervised learning.]{
    An overview of approaches within the three categories of
    self"=supervised learning. 
    Column~(a) lists the names of the models and related references, 
    column~(b) defines the model input, 
    column~(c) defines any corruption of the input or hidden representation, and
    column~(d) defines the target of the pretext task; the pretext task itself
    is described by the overall model category and the main text.
    Notation details are in \cref{footnote:table:pretext}.
    }
    % \begin{minipage}{\textwidth}  
\centering
\resizebox{0.99\textwidth}{!}{
\begin{tabular}{l|c|c|c}
    \toprule
    \textbf{Model} (a) & \textbf{Input} (b) & \textbf{Corruption} (c) & \textbf{Target} (d) \\
    \midrule
    \midrule
    \multicolumn{4}{c}{\textsc{Generative models}} \\
    \midrule
    \midrule
    Audio Word2vec~\parencite{chung_audio_2016}, VQ-VAE~\parencite{oord_neural_2018}     & $X$ &   \textsc{-}    &  $X$  \\ %It has VQ as extra constraint.
    \midrule  
    Speech2Vec~\parencite{chung_speech2vec_2018}, Audio2Vec~\parencite{tagliasacchi_pretraining_2020} - skip-gram    & $X_{[t_1,t_2]}$  &     \textsc{-}   &    $X_{[t_0,t_1]}$,$X_{[t_2,t_3]}$     \\
    \midrule 
    Speech2Vec~\parencite{chung_speech2vec_2018}, Audio2Vec~\parencite{tagliasacchi_pretraining_2020} - cbow    & $X_{[t_0,t_1]}$,$X_{[t_2,t_3]}$   &     \textsc{-}   &    $X_{[t_1,t_2]}$   \\
    \midrule 
    PASE~\parencite{pascual_learning_2019}, PASE+~\parencite{ravanelli_multitask_2020}\textsuperscript{\ref{footnote_review:pase}}       & $X$ &  \textsc{-}   &  Different modalities of $X$  \\
    \midrule 
    APC~\parencite{chung_unsupervised_2019,chung_vectorquantized_2020}         & $X_{[1,t]}$   & \textsc{-}              & $x_{t+c},\, c\geq1$    \\ %VQ-APC is the same as APC
    \midrule
    Speech-XLNet \parencite{song_speechxlnet_2020}     & \multicolumn{2}{c|}{$X_{\mathcal{P}_{t}}$}   &     $x_{i\sim\mathcal{P}^c_{t}}$  \\ %Lee: Use * to represent the permutation, hope it is not strange.
    \midrule  
    DeCoAR~\parencite{ling_deep_2020}     & $X_{[1,t-1]}, X_{[t+k+1,T]}$ & \textsc{-} & $X_{[t,t+k]}$   \\
    \midrule
    Mockingjay~\parencite{liu_mockingjay_2020}, Audio ALBERT~\parencite{chi_audio_2020}, DeCoAR 2.0~\parencite{ling_decoar_2020}   & \multicolumn{2}{c|}{$X_{-[t,t+k]}$}   & $X_{[t,t+k]}$    \\
    \midrule 
    TERA~\parencite{liu_tera_2021}, BMR~\parencite{wang_unsupervised_2020}  & \multicolumn{2}{c|}{$X_{-[t,t+k]}^{-[f,f+j]}$}       & $X$       \\
    \midrule
    pMPC~\parencite{yue_phonetically_2021}      &  \multicolumn{2}{c|}{$X_{-[t,t+k^\prime]}$ ($X_{[t,t+k^\prime]}$ is a phoneme)}       & $X_{[t,t+k^\prime]}$    \\
    \midrule 
    MPE~\parencite{liu_masked_2020} & $X$ &  $Z_{-[t,t+k]}$  & $Z$    \\ %Lee: What is the difference between MPE and NPC??? And I believe it reconstruct the convolutional blocks output (learned target?)
    \midrule
    NPC~\parencite{liu_nonautoregressive_2020}      & $X$  &   $Z_{-[t,t+k]}$  &    $X$   \\ %Lee: I am not 100% sure it is correct. Please check.
    \midrule
    \midrule
    \multicolumn{4}{c}{\textsc{Contrastive models}} \\
    \midrule
    \midrule
    Unspeech \parencite{milde_unspeech_2018}       &   $X_{[t_1,t_2]}$ &   \textsc{-}   &  $X_{[t_0,t_1]}$,$X_{[t_2,t_3]}$ \\
    \midrule 
    CPC~\parencite{oord_representation_2018}, wav2vec \parencite{schneider_wav2vec_2019}, Modified CPC \parencite{riviere_unsupervised_2020}         & $X_{[1,t]}$   &    \textsc{-}           & $z_{t+c},\, c\geq1$   \\ %Modified CPC is the same as CPC
    \midrule 
    Bidirectional CPC \parencite{kawakami_learning_2020}      & $X_{[1,t]}$ or $X_{[t,T]}$ &  \textsc{-}    &    $z_{t+c}$ or $z_{t-c},\, c\geq1$   \\
    \midrule 
    vq-wav2vec \parencite{baevski_vqwav2vec_2020}     &   $X_{[1,t]}$ &   \textsc{-}    &   $q_{t+c},\, c\geq1$   \\ 
    \midrule 
    wav2vec 2.0 \parencite{baevski_wav2vec_2020}, wav2vec-C \parencite{sadhu_wav2vecc_2021}\textsuperscript{\ref{footnote_review:wav2vec-c}}    & $X$             & $Z_{-[t,t+k]}$          & $Q_{[t,t+k]}$ \\
    \midrule 
    w2v-BERT \parencite{chung_w2vbert_2021}     &$X$ &    $Z_{-[t,t+k]}$   &     $Q_{[t,t+k]}$ and $C_{[t,t+k]}$     \\
    \midrule
    Speech SimCLR \parencite{jiang_speech_2021}\footnote{Speech SimCLR targets the latent representation of an augmented version of $X$ using a differently augmented $X$, and vice-versa.}    & \multicolumn{2}{c|}{$\textsc{augment}_1(X)$ and $\textsc{augment}_2(X)$}     &    $\textsc{augment}_2(Z)$ and $\textsc{augment}_1(Z)$   \\ 
    \midrule
    \midrule 
    \multicolumn{4}{c}{\textsc{Predictive models}} \\
    \midrule
    \midrule
    Discrete BERT~\parencite{baevski_vqwav2vec_2020,baevski_effectiveness_2020} \textsuperscript{\ref{footnote_review:discrete-bert}}      &   \multicolumn{2}{c|}{$C_{-[t,t+k]}$}   & $C_{[t,t+k]}$  \\
    \midrule 
    HuBERT \parencite{hsu_hubert_2021}\textsuperscript{\ref{footnote_review:hubert}}, WavLM \parencite{chen_wavlm_2021}\textsuperscript{\ref{footnote_review:wavlm}}  & $X$             & $Z_{-[t,t+k]}$          & $C_{[t,t+k]}$  \\ 
    \midrule
    data2vec \parencite{baevski_data2vec_2022}    & $X$             & $Z_{-[t,t+k]}$          & $\sum_{l}\bar{H}^{(l)}_{[t,t+k]}$  \\ 
    \midrule 
    BEST-RQ \parencite{chiu_selfsupervised_2022}\textsuperscript{\ref{footnote_review:best-rq}}    &  \multicolumn{2}{c|}{$X_{-[t,t+k]}$}      &  $C_{[t,t+k]}$   \\ 
    \midrule 
    \bottomrule
\end{tabular}
}
    
% \end{minipage}
\label{table:pretext}
\end{table}


% \addtocounter{footnote}{1}
\footnotetext{\label{footnote:table:pretext}
$X=\{x_1,x_2,...,x_T\}$ is the input sequence in which $x_t$ can be an
acoustic feature vector (e.g., MFCC, filterbank, or spectrogram features)
or a waveform sample. 
$X_{[t_1:t_2]}$ represents $\{x_{t_1},x_{t_1+1},...,x_{t_2}\}$.
$X_{-[t_1:t_2]}$ represents $X$ in which the segment
$X_{[t_1:t_2]}=\{x_{t_1},x_{t_1+1},...,x_{t_2}\}$ is masked.
$x_{t}^{i}$ represents the $i$-th dimension of $x_t$.
If $x_t$ is a frame in a spectrogram, then the $i$-th dimension corresponds
to a specific frequency bin.
$X^{-[f,f+j]}$ refers to a spectrogram $X$ which is masked along the frequency
axis from the $f$-th to $(f+j)$-th bin. 
% $X_{-[t:t+k]}^{-[f,f+j]}$ refers to a spectrogram $X$ but masked along both time and frequency dimensions. 
% $X^*$ is a temporally permuted version of $X$, that is, the $x_t$ are randomly shuffled to form $X^*$. 
We indicate random temporal permutation of a sequence by indexing it with
the set $\mathcal{P}_t\triangleq\textsc{permute}([0,t])$, where
$\textsc{permute}(\cdot)$ returns a permutation of the given list. 
We indicate data augmentation (e.g., reverberation) by the function
$\textsc{augment}(\cdot)$. Subscripts indicate different augmentations. 
% $X^\prime$ is an augmented version of $X$ (e.g., $X$ with reverberation), while $X^{\prime\prime}$ is $X$ adding distortion different from  $X^\prime$.
$Z$ represents a localized latent representation sequence of $X$. %, and $Z^\prime$ and $Z^{\prime\prime}$ are the latent representation sequences of  $X^\prime$ and $X^{\prime\prime}$, respectively. 
% $Z$ represents a localized latent representation sequence and $Z^\prime$ is the latent representation sequence of $X^\prime$. 
$Z^{(l)}$ is $Z$ at the $l$-th layer of the model used to compute it.
$\bar{H}$ is the contextualized sequence $H$ obtained from an exponential
moving average (EMA) of the model undergoing training with no masking
applied.
$Q$ represents a sequence of quantized learned representations, and $C$ is
a sequence of discrete cluster IDs.
For contrastive models, we specify only positive targets.%
}


\addtocounter{footnote}{1}
\footnotetext{\label{footnote_review:pase}PASE uses multiple pretext tasks, but the authors find that reconstruction is most important.}

\addtocounter{footnote}{1}
\footnotetext{\label{footnote_review:wav2vec-c} wav2vec-C adds reconstruction loss to wav2vec 2.0.}

\addtocounter{footnote}{1}
\footnotetext{\label{footnote_review:discrete-bert}Discrete BERT obtains codes $C$ from vq-wav2vec.}

\addtocounter{footnote}{1}
\footnotetext{\label{footnote_review:hubert}HuBERT is trained first using cluster IDs of the MFCCs as target and subsequently clusters IDs of the model representations from the last iteration.}

\addtocounter{footnote}{1}
\footnotetext{\label{footnote_review:wavlm}WavLM simulates noisy/overlapped speech as inputs.}

\addtocounter{footnote}{1}
\footnotetext{\label{footnote_review:best-rq}BEST-RQ obtains codes $C$ by quantizing acoustic features using a random projection quantizer.}


