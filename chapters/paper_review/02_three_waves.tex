%!TEX root = ../thesis.tex

\section{Historical context of representation learning}

\label{sec:thirdwave}

In this section we present the historical background of the current surge in
self"=supervised representation learning methods in the context of two previous
waves of research work in the 1990s and 2000s. The discussed approaches go
beyond speech to describe the overall landscape of machine learning development during the
past few decades.

\subsection{Clustering and mixture models}

Initial research in learning latent speech and audio representations involved
simple models in which the training data likelihood was optimized directly
or via the expectation--maximization (EM) algorithm.

Early work used simple clustering methods. For example, in work such
as \parencite{rabiner_considerations_1979,wilpon_modified_1985}, word patterns were clustered
semi-automatically using techniques such as $k$-means, after which isolated words
were recognized by finding the training cluster closest to 
  the test data.

Through time, modeling techniques improved such that subword units were
represented by Gaussian mixture models (GMMs) \parencite{gauvain_maximum_1994}, which facilitated
the modeling of more variability in the input data. GMMs were first built for
context-independent phonemes; state-clustering 
algorithms~\parencite{young_state_1994} then resulted in GMMs for context-dependent phonemes. 
Each latent component of these mixture models acted as a template of a
prototypical speech frame, 
making it difficult to handle 
large volumes of data with diverse characteristics. 
Furthermore, dynamical models like hidden Markov models (HMMs)~\parencite{bell_adaptation_2021}
allowed for the processing of continuous speech rather than just isolated word
recognition. These generative GMM and HMM models were trained by maximizing the
likelihood of data given the model, which could be accomplished in either an
unsupervised or a supervised manner.

Another line of research focused on extracting speech features from generative
models. The main objective here was to render the knowledge learned by generative
models accessible to discriminative downstream classifiers, or to map
variable-length sequences to fixed-length representations. Feature vectors
were derived from the parameters of trained GMM models. In the case of {\em
Fisher vectors}, the features were the normalized gradients of the
log-likelihood with respect to the model parameters (mixture weights, means, and
variances) of the Gaussian mixtures. An extension of this approach (likelihood
ratio score space) used the derivative of the log-likelihood ratio of two
models, e.g., a background model and a foreground model. Examples of their use
in speech processing include speech recognition~\parencite{smith_speech_2001,venkataramani_support_2003} and
speaker recognition~\parencite{wan_svmsvm_2003}. Subsequent techniques in speaker and language
verification~\parencite{dehak_frontend_2011,dehak_language_2011} similarly extracted parameters
(concatenated means) from trained background GMMs as representations that were
then combined with low-rank projections of speaker/session- or language-specific
vectors.



\subsection{Stacked neural models}
\label{subsec:stack}

More recently, representation learning has seen a shift of focus towards neural models,
which, compared to GMMs and HMMs, offer distributed representations with more
capacity to model diverse input signals into efficient latent binary codes. 
Examples of early techniques include restricted Boltzmann machines
(RBM)~\parencite{hinton_reducing_2006}, denoising autoencoders~\parencite{vincent_stacked_2010}, noise contrastive
estimation (NCE)~\parencite{gutmann_noisecontrastive_2010}, sparse coding~\parencite{olshausen_emergence_1996,
lee_efficient_2006, sivaram_sparse_2010}, and energy-based methods~\parencite{ranzato_unified_2007}.
Many of these techniques have also been applied to CV and NLP problems, which
provided inspiration for their application to speech.

Higher-capacity neural models were achieved by stacking several neural network
layers to build progressively higher-level concept representations.
However, these deeper networks also increased the training complexities. For
example, approximate training methods such as contrastive divergence
\parencite{hinton_training_2002} were a practical technique to streamline RBM training. 
Furthermore, deep networks had non-convex objective functions, which
often resulted in long training times compared to GMMs, which are trained
using full batches instead of mini-batch learning.

\subsection{Learning through pretext task optimization}

A more recent trend is learning networks that map the input to desired
representations by solving a \textit{pretext} task. Such studies have several
characteristics:
(1)~All layers are trained end-to-end to optimize a single pretext task instead
of relying on layer-wise pre"=training
(2)~Past stacked networks typically had only a few layers, but 
very deep networks with more than ten layers are now common.
(3)~It is common to evaluate a representation model on a wide range of tasks.
For example, in NLP, a representation model is usually assessed on GLUE,
which comprises nine tasks~\parencite{wang_glue_2018}, whereas in speech, a representation model can be
evaluated on SUPERB, which comprises ten tasks~\parencite{yang_superb_2021}, 
as described in detail in \cref{sec:benchmark}.

The cornerstone of this third wave is the design of a pretext task, which
allows the model to efficiently leverage knowledge from unlabeled data.
The pretext task should be challenging enough for the model to learn high-level
abstract representations and 
% not be too amiable to exploit low-level shortcuts.
  not be so easy as to encourage the exploitation of low-level shortcuts.  % AMH: check
Early breakthroughs included end-to-end learning of deep neural architectures
via pretext tasks for restoring the true color of black-and-white
images~\parencite{zhang_colorful_2016}, joint learning of latent representations and their
cluster assignments~\parencite{caron_deep_2018}, and the prediction of the relative positions of
image patches~\parencite{doersch_unsupervised_2015}. Other popular approaches include variational
autoencoders (VAEs)~\parencite{kingma_autoencoding_2014, rezende_stochastic_2014}. While typical autoencoders learn data
representations using unsupervised objectives by reconstructing the input
after passing it through an information bottleneck, VAEs estimate a neural model of a probability density function (pdf) that approximates the unknown “true” distribution of the observed data, for which we only have access to independently identically distributed (iid) samples. It is also important to mention dynamical VAEs \parencite{girin_dynamical_2021}, which is an extension of VAE for sequential data such as speech.

In the SSL context, a pretext task related to autoencoding is to generate
an object from its partial information. Such tasks are widely used in NLP, for
example, using 
the previous tokens 
in a sentence to predict the next token such as in
ELMo~\parencite{peters_deep_2018}, the GPT
series~\parencite{radford_language_2019}, and 
Megatron~\parencite{shoeybi_megatronlm_2020}, or predicting the masked tokens in a
sentence such as with the bidirectional encoder representations from Transformers
(BERT) series~\parencite{devlin_bert_2018,liu_roberta_2019}. 
Another common pretext task in the third wave is contrastive
learning~\parencite{oord_representation_2018}, in which a model learns to identify a
target instance from a set of negative samples. 
This approach has become especially popular in the
CV context~\parencite{chen_simple_2020,he_momentum_2020,chen_improved_2020,caron_unsupervised_2020}.
In this survey, we will mainly focus on techniques for pretext task
optimization for speech processing, and discuss these techniques in detail
in \cref{sec:approach}.


\subsection{Other related work}
A closely related area of research that is not covered in this review is
semi"=supervised pre"=training methods such as pseudo"=labeling (that is,
self-training). Pseudo"=labeling (PL) relies on a supervised teacher model to
label a large volume of speech-only data, which is then used to augment the
initial labeled data to train a student model~\parencite{kemp_unsupervised_1999, lamel_lightly_2002,
ma_unsupervised_2006, parthasarathi_lessons_2019}. PL has been successful and widely adopted in the
speech community since the 1990s. Other proposed variations of PL include
augmenting speech-only data with noise to improve robustness, iterating
over the PL process to improve teacher labeling quality, and training student
models with more parameters than their original teachers to capture the
complexities in vastly larger speech-only data~\parencite{park_improved_2020,
xu_iterative_2020, xiao_scaling_2021}. 
Both SSL and PL leverage unlabeled speech-only data.
One distinguishing factor in PL is the utilization of supervised data for a
specific task during model pre"=training, which limits the model's focus to
a single (or at best a few) downstream tasks. 
SSL, in turn, is an attempt to learn task-agnostic representations to benefit
a wide range of tasks. 

Transfer learning (TL) is another closely related area of research for
pre"=training speech models. TL transfers knowledge captured by models
trained on one task to different but related tasks~\parencite{caruana_multitask_1997}. 
The past few decades have seen active research on TL and its
extension to multitask learning for more general representations. 
Multilingual and cross-lingual supervised models have proven superior in
low-resource speech recognition tasks~\parencite{cui_multilingual_2015}.
SSL can be regarded as a type of TL because knowledge learned from pre"=training
is used for different downstream tasks.
This survey paper focuses on SSL, and not all TL technologies for speech. 
One survey indeed addresses TL for speech processing~\parencite{bell_adaptation_2021}
but does not include current SSL technologies for speech.  
