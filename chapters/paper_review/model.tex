%!TEX root = ../thesis.tex

\begin{table*}[t]
\caption{
The models are sorted according to publication date of the first version on arXiv which might differ from the year indicated by the citation. 
\textbc{{reg}} means that the model uses a regression loss, 
\textbc{{clf}} that it uses a classification loss,
\textbc{{con}} that it uses a contrastive loss, 
\textbc{{msk}} that it uses masking to define the target region, 
\textbc{{ltg}} that it uses a learned target, 
\textbc{{qtz}} that it uses a quantized target, and
\textbc{{emb}} that it embeds a variable length segment.
%Lee: I think contrastive always have learned target. Am I correct???
}
\label{tab:model-taxonomy}
\begin{center}
\renewcommand{\arraystretch}{1.2}
\begin{tabular}{ l l | c c c | c c c | c } 
\toprule
\multicolumn{2}{c}{} & 
\multicolumn{3}{c}{\small\textsc{Loss}} & 
\multicolumn{3}{c}{\small\textsc{Target}} &
 \\
\textbc{{model}} &
\textbc{{pub. date}} &
\textbc{{reg}} & 
\textbc{{clf}} & 
\textbc{{con}} &
\textbc{{msk}} &
\textbc{{ltg}} &  
\textbc{{qtz}} & 
\textbc{{emb}} \\
\midrule
%                                                         REG      CLF      CON      MSK      LTG      QTZ      EMB  

Audio Word2vec \parencite{chung_audio_2016}      & 2016 Mar. & \cmark & \xmark & \xmark & \xmark & \xmark & \xmark & \cmark \\

VQ-VAE \parencite{oord_neural_2018} & 2017 Nov. & \cmark & \xmark & \xmark & \xmark & \xmark & \cmark & \xmark \\ %2 Nov 2017 

Speech2Vec \parencite{chung_speech2vec_2018}     & 2018 Mar. & \cmark & \xmark & \xmark & \xmark & \xmark & \xmark & \cmark \\
Unspeech \parencite{milde_unspeech_2018}         & 2018 Apr. & \xmark & \xmark & \cmark & \xmark & \xmark & \xmark & \cmark \\

CPC \parencite{oord_representation_2018}         & 2018 Jul. & \xmark & \xmark & \cmark & \xmark & \cmark & \xmark & \xmark \\

PASE \parencite{pascual_learning_2019}                   & 2019 Apr. & \cmark & \xmark & \cmark & \xmark & \cmark & \xmark & \xmark \\ % 6/4

APC \parencite{chung_unsupervised_2019}          & 2019 Oct. & \cmark & \xmark & \xmark & \xmark & \xmark & \xmark & \xmark \\ % 5/4

wav2vec \parencite{schneider_wav2vec_2019}       & 2019 Apr. & \xmark & \xmark & \cmark & \xmark & \cmark & \xmark & \xmark \\ % 11/4

vq-wav2vec \parencite{baevski_vqwav2vec_2020} & 2019 Oct. & \xmark & \xmark & \cmark & \xmark & \cmark & \cmark & \xmark \\ %12 Oct 2019 %Lee: vq-wav2vec is CON or CLF??? %Lee: Discrete BERT use masking, but vq-wav2vec does not It is CON, not CLF.

Discrete BERT \parencite{baevski_vqwav2vec_2020,baevski_effectiveness_2020} & 2019 Oct. & \xmark & \cmark & \cmark & \cmark & \cmark & \cmark & \xmark \\ %12 Oct 2019 %10 Nov 2019 %In the paper of vq-wav2vec, it also includes the approach of Discrete BERT. I know this is tricky ....

Mockingjay  \parencite{liu_mockingjay_2020}      & 2019 Oct. & \cmark & \xmark & \xmark & \cmark & \xmark & \xmark & \xmark \\ % 25/10-19


DeCoAR \parencite{ling_deep_2020}                & 2019 Dec. & \cmark & \xmark & \xmark & \cmark & \xmark & \xmark & \xmark \\ % 3/12-19

PASE+ \parencite{ravanelli_multitask_2020}                & 2020 Jan. & \cmark & \xmark & \cmark & \xmark & \cmark & \xmark & \xmark \\ % 25/1 % I think contrastive always have learned target.

Bidir CPC \parencite{kawakami_learning_2020} & 2020 Jan. & \xmark & \xmark & \cmark & \xmark & \cmark & \xmark & \xmark \\% 29/1

Modified CPC \parencite{riviere_unsupervised_2020} & 2020 Feb. & \xmark & \xmark & \cmark & \xmark & \cmark & \xmark & \xmark \\ % 7/2

VQ-APC \parencite{chung_vectorquantized_2020}  & 2020 May  & \cmark & \xmark & \xmark & \xmark & \xmark & \cmark & \xmark \\ %17/05

AALBERT \parencite{chi_audio_2020}  & 2019 May & \cmark & \xmark & \xmark & \cmark & \xmark & \xmark & \xmark \\ %18 May 2020

speech-XLNet \parencite{song_speechxlnet_2020} & 2019 Oct. & \cmark & \xmark & \xmark & \xmark & \xmark & \xmark & \xmark \\%23 Oct 2019 %Lee: It is speech version of XLNet. I am not sure we should consider this approach has mask or not.


wav2vec 2.0 \parencite{baevski_wav2vec_2020}     & 2020 Jun. & \xmark & \xmark & \cmark & \cmark & \cmark & \cmark & \xmark \\ % 20/6
%Lee: In my point of view, wav2vec 2.0 has learned target. The target is from the CNN encoder.

MPE \parencite{liu_masked_2020} & 2020 May & \cmark & \xmark & \xmark & \cmark & \cmark & \xmark & \xmark \\ % 25 May 2020 %Lee: I think it has learned target. But please check.

TERA \parencite{liu_tera_2021}                   & 2020 Jul. & \cmark & \xmark & \xmark & \cmark & \xmark & \xmark & \xmark \\ % 12/7

Speech SimCLR \parencite{jiang_speech_2021} &  2020 Oct. & \cmark & \xmark & \cmark & \cmark & \cmark & \xmark & \xmark \\ %Lee: It is TERA+ SIMCLR. I considered it has learned target because contrastive always have learned target.
%27 Oct 2020

NPC \parencite{liu_nonautoregressive_2020}                     & 2020 Nov. & \cmark & \xmark & \xmark & \cmark & \xmark & \cmark & \xmark \\ % 1/11

DeCoAR 2.0 \parencite{ling_decoar_2020}          & 2020 Dec. & \cmark & \xmark & \xmark & \cmark & \xmark & \cmark & \xmark \\ % 11/12 %Lee:  DeCoAR 2.0 has VQ layer although it is not used as learning target.

wav2vec-C \parencite{sadhu_wav2vecc_2021} & 2021 Mar. & \cmark & \xmark & \cmark & \cmark & \cmark & \cmark & \xmark \\ %9 Mar 2021 %Lee:  consistency loss is reconstruction

HuBERT \parencite{hsu_hubert_2021}               & 2021 Jun. & \xmark & \cmark & \xmark & \cmark & \cmark & \cmark & \xmark \\ % 14/6

W2v-bert \parencite{chung_w2vbert_2021} & 2021 Aug & \xmark & \cmark & \cmark & \cmark & \cmark & \cmark & \xmark \\
%7 Aug 2021

pMPC \parencite{yue_phonetically_2021}  & 2021 Aug. & \cmark & \xmark & \xmark & \cmark & \xmark & \xmark & \xmark \\ %30 August – 3 September, 2021 (I do not find arXiv version.)

WavLM \parencite{chen_wavlm_2021} & 2021 Oct. & \xmark & \cmark & \xmark & \cmark & \cmark & \cmark & \xmark \\ %26 Oct 2021

data2vec \parencite{baevski_data2vec_2022}     & 2022 Jan. & \cmark & \xmark & \xmark & \cmark & \cmark & \xmark & \xmark \\ % 20/1

BEST-RQ \parencite{chiu_selfsupervised_2022} & 2022 Feb.  & \xmark & \cmark & \xmark & \cmark & \xmark & \cmark & \xmark \\ %Lee: no learned target. The target is random ...
% 3 Feb 2022

\bottomrule
\end{tabular}
\end{center}
\end{table*}


\begin{comment}
\begin{table*}[h!]
\caption{
The models are sorted according to the publication date of the first version on arXiv, which might differ from the year indicated by the citation. 
\textbc{{reg}} means that the model uses a regression loss, 
\textbc{{clf}} that it uses a classification loss,
\textbc{{con}} that it uses a contrastive loss, 
\textbc{{msk}} that it uses masking to define the target region, 
\textbc{{ltg}} that it uses a learned target, %lasseborgholt: We could consider using the inverse of this attribute. I.e., that the model uses the input as taget and belongs in the generative category.
\textbc{{qtz}} that it uses a quantized target, and
\textbc{{emb}} that it embeds a variable-length segment
}
\label{tab:model-taxonomy}
\begin{center}
\renewcommand{\arraystretch}{1.2}
\begin{tabular}{ l l | c c c | c c c | c } 
\toprule
\multicolumn{2}{c}{} & 
\multicolumn{3}{c}{\small\textsc{Loss}} & 
\multicolumn{3}{c}{\small\textsc{Target}} &
 \\
\textbc{{model}} &
\textbc{{pub. date}} &
\textbc{{reg}} & 
\textbc{{clf}} & 
\textbc{{con}} &
\textbc{{msk}} &
\textbc{{ltg}} &  
\textbc{{qtz}} & 
\textbc{{emb}} \\
\midrule
%                                                         REG      CLF      CON      MSK      LTG      QTZ      EMB  
Audio Word2vec \parencite{chung_audio_2016}      & 2016 Mar. & \cmark & \xmark & \xmark & \xmark & \xmark & \xmark & \cmark \\
Speech2Vec \parencite{chung_speech2vec_2018}     & 2018 Mar. & \cmark & \xmark & \xmark & \xmark & \xmark & \xmark & \cmark \\
% Unspeech \parencite{milde_unspeech_2018}         & 2018 Apr. & \xmark & \xmark & \cmark & \xmark & \xmark & \xmark & \cmark \\
CPC \parencite{oord_representation_2018}         & 2018 Jul. & \xmark & \xmark & \cmark & \xmark & \cmark & \xmark & \xmark \\

PASE \parencite{pascual_learning_2019}                   & 2019 Apr. & \cmark & \xmark & \cmark & \xmark & \xmark & \xmark & \xmark \\ % 6/4

APC \parencite{chung_unsupervised_2019}          & 2019 Oct. & \cmark & \xmark & \xmark & \xmark & \xmark & \xmark & \xmark \\ % 5/4
wav2vec \parencite{schneider_wav2vec_2019}       & 2019 Apr. & \xmark & \xmark & \cmark & \xmark & \cmark & \xmark & \xmark \\ % 11/4
%VQ-APC \parencite{chung_vectorquantized_2020}             & 2020 May  & \xmark & \xmark & \xmark & \xmark & \xmark & \xmark & \xmark \\ 
Mockingjay  \parencite{liu_mockingjay_2020}      & 2019 Oct. & \cmark & \xmark & \xmark & \cmark & \xmark & \xmark & \xmark \\ % 25/10-19
%DeCoAR \parencite{ling_deep_2020}                & 2019 Dec. & \xmark & \xmark & \xmark & \xmark & \xmark & \xmark & \xmark \\ % 3/12-19
%PASE+ \parencite{ravanelli_multitask_2020}                & 2020 Jan. & \xmark & \xmark & \xmark & \xmark & \xmark & \xmark & \xmark \\ % 25/1
wav2vec 2.0 \parencite{baevski_wav2vec_2020}     & 2020 Jun. & \xmark & \xmark & \cmark & \cmark & \xmark & \cmark & \xmark \\ % 20/6
%TERA \parencite{liu_tera_2021}                   & 2021 Jul. & \xmark & \xmark & \xmark & \xmark & \xmark & \xmark & \xmark \\ % 12/7

NPC \parencite{liu_nonautoregressive_2020}             & 2020 Nov. & \cmark & \xmark & \xmark & \cmark & \xmark & \xmark & \xmark \\ % 1/11

DeCoAR 2.0 \parencite{ling_decoar_2020}          & 2020 Dec. & \cmark & \xmark & \xmark & \cmark & \xmark & \xmark & \xmark \\ % 11/12
HuBERT \parencite{hsu_hubert_2021}               & 2021 Jun. & \xmark & \cmark & \xmark & \cmark & \cmark & \cmark & \xmark \\ % 14/6
data2vec \parencite{baevski_data2vec_2022}     & 2022 Jan. & \cmark & \xmark & \xmark & \cmark & \cmark & \xmark & \xmark \\ % 20/1
\bottomrule
\end{tabular}
\end{center}
\end{table*}

\end{comment}
