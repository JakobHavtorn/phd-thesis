%!TEX root = ../thesis.tex

\chapter[conclusions and outlook]{Conclusions and outlook}\label{chp:conclusion}
% ~3 pages

In the introductory \cref{chp:introduction} we laid out how, through human history, progress-driving technological development has always introduced new challenges and risks of error whether via misuse, misunderstanding or inherent limitations. We emphasized that machine learning comes with these same types of risk, in some cases amplifying their impacts, and argued that uncertainty estimation is a key component in ensuring that systems are safe and reliable in practice. We provided background for the research project by introducing the healthtech company Corti, with which with project has been defined and carried out, and the motivational cases of recognition of stroke cases in emergency calls (\cref{subsec:motivation-stroke-recognition}) and automated medical coding (\cref{subsec:motivation-medical-coding}). We described how machine learning systems could be designed to aid healthcare professionals with the tasks and gave examples of potential sources of uncertainty and of how correctly quantifying uncertainty might improve the usefulness of such systems. Finally, we connected machine learning reliability (\cref{sec:machine-learning-reliability}) and model calibration (\cref{subsec:model-calibration}) and defined the aleatoric, epistemic, and predictive types of uncertainty (\cref{subsec:understanding-uncertainty}).

\Cref{chp:technical-background} provided technical background only covered briefly by the individual studies. We first introduced uncertainty as a concept in the context of information and probability theory \cref{sec:uncertainty-information-theory}. Then, we defined the task of out-of-distribution detection and reviewed existing work on the problem \cref{sec:out-of-distribution-detection}. Finally, we provided technical background for variational autoencoders \cref{sec:variational-autoencoders}.

\dots
