%!TEX root = ../thesis.tex

\chapter[conclusions and outlook]{Conclusions and outlook}\label{chp:conclusion}
% ~3 pages

In the introductory \cref{chp:introduction} we laid out how progress-driving technological development has always come with new challenges and risks of error - whether via misuse, misunderstanding or inherent limitations. We emphasized that machine learning comes with these same types of risk, in some cases amplifying their impacts, and argued that uncertainty estimation is a key component in ensuring that systems are safe and reliable in practice. We also provided the background for the research project by introducing the healthtech company Corti, with which with project has been defined and carried out, and the motivational cases of recognition of stroke cases in emergency calls (\cref{subsec:motivation-stroke-recognition}) and automated medical coding (\cref{subsec:motivation-medical-coding}). We described how machine learning systems could be designed to aid healthcare professionals with these tasks and gave examples of potential sources of uncertainty and of how correctly quantifying it might improve the usefulness of such systems. Finally, we drew connections between machine learning reliability (\cref{sec:machine-learning-reliability}) and model calibration (\cref{subsec:model-calibration}) and defined the aleatoric, epistemic, and predictive types of uncertainty (\cref{subsec:understanding-uncertainty}).


\Cref{chp:technical-background} provided technical background only covered briefly by the individual studies. We first introduced uncertainty as a concept in the context of information and probability theory (\cref{sec:uncertainty-information-theory}). Then, we defined the task of out-of-distribution detection and reviewed existing work on the problem (\cref{sec:out-of-distribution-detection}). Finally, we provided technical background for variational autoencoders (\cref{sec:variational-autoencoders}).


In \cref{chp:paper-hierarchical} 

In \cref{chp:paper-modelagnostic} 

In \cref{chp:paper-benchmarking} 
% In chapter 8, stochastic and deterministic generative models were benchmarked in terms of model likelihood. In addition, the chapter presented a hierarchical latent variable model (LVM) for speech inspired by the Clockwork VAE [238]. In contrast to self-supervised models, LVMs have not yet seen the same success for representation learning. As such, they offer little guidance in terms of alleviating the need for labeled data, as discussed in section 1.2.1.

% However, since LVMs learn a distribution over the training data, they are forced to encode all aspects of it. While this may be redundant for some tasks, others benefit from a broad variety of features. A hierarchical model that operates on multiple temporal scales, such as the Clockwork VAE, offers a natural way to encode different feature categories. Pronunciation might be learned at lower layers, speaker identity at the upper layers, and semantic features in between. Although some work has successfully separated speaker identity from content [114], models that can learn a deep hierarchy of features for speech remains an open challenge. The model presented in chapter 8 represents an effort to reignite this area of research.

In \cref{chp:paper-brief}

% Chapter 6 presented an overview of unsupervised neural speech representation learning. As emphasized in the overview, the wav2vec 2.0 framework [13], and masked pre-training in general, represent a breakthrough in low-resource speech recognition. The challenge associated with obtaining training data for conversational speech recognition, as discussed in section 1.2.1, has become much smaller. Even when no labeled in-domain data is available, this framework offers a viable solution [112].

% The general idea for masked pre-training may seem simple; reconstruct masked parts of the input or another representation, given context. However, this methodology aligns very well with the general definition of semantics presented in section 1.4: "semantic properties of a lexical item are fully reflected in appropriate aspects of the relations it contracts with actual and potential contexts" ([63], p. 1). Thus, from a philosophical point-of-view, it may be difficult to see how the field should move on from here. Of course, context is more than just the neighboring words in a speech segment. The next wave of representation learning for speech incorporates multiple modalities, such as video [244, 245] or text [18]. Furthermore, how to construct targets for pre-training these models and how that affects the learned features is an avenue that warrants more research. This is of particular interest in the speech domain, where the input codes for speaker identity and emotional state, as well as the semantic content of the utterance.


\cref{chp:paper-automated}

\cref{chp:paper-retrospective}


