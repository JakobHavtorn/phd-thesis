%!TEX root = ../thesis.tex

\chapter[introduction]{Introduction}\label{chp:introduction}

% ~9 pages

Uncertainty is a fundamental part of human experience. Understanding and quantifying uncertainty is crucial in making informed decisions, drawing meaningful conclusions, and assessing the reliability of predictions made from observations. Yet, capturing it in a mathematical model is a difficult task. 

Uncertainty estimation has been a central topic in the field of statistics. It has traditionally been addressed through methods like confidence intervals, hypothesis testing, and Bayesian inference. Confidence intervals provide a range of plausible values for a parameter, while hypothesis testing evaluates the significance of differences between groups or variables \cite{blitzstein_introduction_2019}. Bayesian inference, on the other hand, offers a powerful framework to incorporate prior knowledge and update beliefs based on observed data using probability theory. These approaches have been extensively studied and applied in various domains, from social sciences to natural sciences, to improve the robustness and generalizability of statistical analyses \cite{gelman_bayesian_2013}. However, the widespread adoption in diverse applications of machine learning, and in particular deep neural networks, has posed new challenges in dealing with uncertainty. 

Machine learning algorithms are designed to learn patterns from data and make predictions or decisions based on those patterns. Since many applications of machine learning are concerned with real-world phenomena, it is important to understand the uncertainty associated with the predictions made by such a model. For instance, in medical diagnosis, knowing the uncertainty in a model's prediction is crucial when the consequences of a false positive or false negative can be significant. In speech recognition, knowing that a certain word was hard to understand for the model can help avoid misinterpretation of the transcribed text.

Speech processing is 

% The performance of machine learning models is highly dependent on the quality and quantity of data used for training. In many cases, the data used to train machine learning models are collected from the real world, and therefore, may contain biases and errors which models may learn to mirror. In addition, the complexity of modern machine learning models makes it difficult to understand how they arrive at their predictions. This is especially true for deep neural networks, which are often referred to as black-box models. The lack of interpretability makes it difficult to assess the reliability of the model's predictions.

% Modern speech processing relies on high-performance parallel computing [40, 157], large volumes of data [136, 213], and years of innovation in model design [109, 266].

% 40: High performance convolutional neural networks for document processing \cite{chellapilla_high_2006}
% 157: Imagenet classification with deep convolutional neural networks \cite{krizhevsky_imagenet_2012}
% 136: Libri-light: A benchmark for asr with limited or no supervision \cite{kahn_libri-light_2020}
% 213: Librispeech: an ASR corpus based on public domain audio books \cite{panayotov_librispeech_2015}
% 109:  Long short-term memory \cite{hochreiter_long_1997}
% 266: Attention Is All You Need \cite{vaswani_attention_2017}


The research in this thesis covers a number of different modelling approaches and tasks, including 


The research done over the course of the project at hand covers a varied set of topics. One group of the produced studies is concerned with generative latent variable models and their applications to uncertainty estimation and speech modelling \cite{havtorn_hierarchical_2021,havtorn_benchmarking_2022,bergamin_modelagnostic_2022}. Another group of studies are concerned with self-supervised speech representation learning and automatic speech recognition \cite{borgholt_scaling_2021,borgholt_we_2021,mohamed_selfsupervised_2022,borgholt_brief_2022}. A final group deals with applications within the medical domain including recognition of stroke cases in calls to medical helplines \cite{wenstrup_retrospective_2023} and medical coding of clinical notes \cite{edin_automated_2023}.

This first part of the thesis ties together these different studies by providing a high-level overview of the research topic and the main contributions of the thesis.



\subsection{Motivation: Speech recognition}


\subsection{Motivation: Automated medical coding}


\subsection{Challenges in uncertainty estimation}




General challenges in uncertainty estimation:
\begin{enumerate}
    \item \textbf{Accuracy}: The uncertainty estimation method should be accurate.
    \item \textbf{Interpretability}: The uncertainty estimation method should be interpretable.
    \item \textbf{Robustness}: The uncertainty estimation method should be robust to adversarial attacks.
    \item \textbf{Applicability}: The uncertainty estimation method should be applicable to a wide range of models without requiring extensive modifications.
    \item \textbf{Efficiency}: The uncertainty estimation method should be computationally efficient at training and prediction time.
    \item \textbf{Scalability}: The uncertainty estimation method should scale to large datasets.
\end{enumerate}


\begin{enumerate}
    \item How can we learn representations that can form the basis of both prediction and uncertainty estimation?
    \item 
\end{enumerate}



\subsection{Scope of the thesis}




\iffalse




Uncertainty is a fundamental part of human experience. Yet, capturing it in a mathematical model is a difficult task. Historically, uncertainty estimation has been a central topic in the field of statistics. While classical statistical methods have provided valuable insights, they often assume well-defined probability distributions and may not fully capture the complexities of uncertainty in real-world scenarios. Moreover, they tend to rely heavily on parametric assumptions, which might not hold true for complex data with high-dimensional features and non-linear relationships. Therefore, in recent years, researchers have turned their attention towards more flexible and data-driven approaches to model uncertainty in various domains.

In the field of machine learning, uncertainty is often represented by a probability distribution. The most common approach is to use Bayesian methods, where uncertainty is captured by posterior distributions over model parameters and predictions. Bayesian neural networks, for instance, offer a powerful framework to model uncertainty in deep learning architectures. By incorporating prior beliefs and updating them based on observed data, these networks can produce probabilistic predictions that provide a measure of uncertainty. This is particularly useful in applications where high-confidence predictions are required, and the consequences of errors can be significant.

Recently, another promising approach to uncertainty estimation in machine learning is the use of Monte Carlo Dropout (MC Dropout). MC Dropout leverages the idea of dropout regularization, originally employed during training to prevent overfitting. During prediction, MC Dropout takes multiple forward passes through the network with dropout enabled, which leads to obtaining a distribution of outputs for each input sample. The variance among these sampled predictions serves as a measure of uncertainty. MC Dropout has shown remarkable success in various tasks such as image classification, object detection, and natural language processing. It is computationally efficient and can be easily incorporated into existing deep learning architectures, making it an attractive choice for uncertainty estimation.

Furthermore, there has been a surge of interest in ensemble methods for uncertainty estimation. Ensembles combine the predictions of multiple models to obtain a more robust and calibrated uncertainty measure. Bagging and boosting techniques, which have been widely used in the field of statistics, have found their way into the realm of machine learning for uncertainty estimation as well. By training multiple models with different initializations or using diverse learning algorithms, ensembles can capture different sources of uncertainty, thereby providing a comprehensive assessment of the overall uncertainty in the predictions.

Despite the progress made in uncertainty estimation for machine learning models, challenges still remain. One key concern is the interpretability of uncertainty measures. While probabilistic outputs are more informative than point estimates, effectively communicating uncertainty to end-users and decision-makers is not trivial. Developing visualization techniques and intuitive explanations for uncertainty is an ongoing area of research. Additionally, quantifying uncertainty in high-dimensional data or complex models with massive amounts of parameters requires careful consideration and efficient computational strategies.

In conclusion, uncertainty estimation is a crucial aspect of machine learning models, enabling them to make more informed and reliable predictions. Researchers continue to explore innovative techniques to better represent and quantify uncertainty, allowing for safer and more trustworthy deployment of machine learning systems in real-world scenarios.





Uncertainty is a fundamental part of human experience. Yet, capturing it in a mathematical model is a difficult task. Historically, uncertainty estimation has been a central topic in the field of statistics. Understanding and quantifying uncertainty is crucial in making informed decisions, drawing meaningful conclusions, and assessing the reliability of predictions and inferences derived from data.


In the field of statistics, uncertainty is traditionally addressed through methods like confidence intervals, hypothesis testing, and Bayesian inference. Confidence intervals provide a range of plausible values for a parameter, while hypothesis testing evaluates the significance of differences between groups or variables. Bayesian inference, on the other hand, offers a powerful framework to incorporate prior knowledge and update beliefs based on observed data using probability theory. These approaches have been extensively studied and applied in various domains, from social sciences to natural sciences, to improve the robustness and generalizability of statistical analyses.
However, the advent of machine learning and its widespread adoption in diverse applications has posed new challenges and opportunities in dealing with uncertainty. 

Machine learning algorithms are designed to learn patterns from data and make predictions or decisions based on that knowledge. In many cases, it is not enough to provide a single deterministic prediction; it is equally important to convey the level of uncertainty associated with the prediction. For instance, in medical diagnosis, knowing the uncertainty in a model's prediction is crucial when the consequences of a false positive or false negative can be significant. In autonomous driving systems, understanding uncertainty becomes essential to ensure the safety of passengers and pedestrians.

To address the uncertainty challenge in machine learning, researchers have developed various techniques to represent and quantify uncertainty. One of the common approaches is the use of probabilistic models, where uncertainty is captured in the form of probability distributions. Instead of providing a single point prediction, probabilistic models generate a range of possible outcomes along with their associated probabilities. Bayesian neural networks, Gaussian processes, and variational autoencoders are some examples of such models that have gained prominence in the quest for uncertainty-aware machine learning.

Another avenue for dealing with uncertainty in machine learning is through ensemble methods. Ensemble learning combines multiple models to improve predictive accuracy and can also provide uncertainty estimates. By aggregating the outputs of diverse models, ensemble methods can identify situations where predictions are consistent among the models, leading to higher confidence, and cases where the models disagree, indicating higher uncertainty.

As the intersection of statistics and machine learning grows increasingly important, the integration of these two domains has become an active area of research. The incorporation of probabilistic modeling and Bayesian approaches into machine learning methods opens up new possibilities for developing more robust, interpretable, and calibrated models. Moreover, the synergy of statistical techniques with deep learning, reinforcement learning, and other advanced machine learning paradigms has the potential to push the boundaries of uncertainty quantification and application even further.

In this thesis, we aim to explore and contribute to the field of uncertainty in statistics and machine learning. We will investigate state-of-the-art methods for uncertainty estimation, delve into their theoretical underpinnings, and evaluate their performance in various real-world scenarios. By gaining a deeper understanding of uncertainty and its implications, we aspire to pave the way for more reliable, trustworthy, and accountable AI systems in the future.

In the following chapters, we will begin by reviewing the relevant literature on uncertainty in both statistics and machine learning, providing a comprehensive overview of the current landscape. We will then present our proposed methodologies and empirical studies, illustrating how uncertainty quantification can enhance decision-making processes and mitigate risks in practical applications. Finally, we will conclude with a discussion of our findings, potential limitations, and future research directions in this exciting and rapidly evolving field.

\fi
