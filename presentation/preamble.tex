% !TEX root = presentation.tex
% !BIB program = biber
% !TEX program = xelatex

% Default packages
\usepackage[english]{babel}
\usepackage[T1]{fontenc}
\usepackage{adjustbox}
\usepackage{booktabs}
\usepackage{booktabs}
\usepackage{caption}
\usepackage{csquotes}
\usepackage{enumerate}
\usepackage{makecell}
\usepackage{multicol}
\usepackage{multicol}
\usepackage{multirow}
\usepackage{rotating}
\usepackage{siunitx}
\usepackage{siunitx}
\usepackage{subcaption}
\usepackage{tabularx,colortbl}
\usepackage{pgfpages}

% Bibliography
\usepackage[
    backend=biber,
    natbib=true, 
    style=numeric,
    abbreviate=false,
    dateabbrev=false,
    maxbibnames=14,
    urldate=long,
    date=year,
    url=false,
    backref=true,
    sortcites
]{biblatex}
\addbibresource{../bibliography/zotero-references.bib}
\setbeamertemplate{bibliography item}{\insertbiblabel}
\renewcommand*{\bibfont}{\normalfont\scriptsize}  % make bibliography font smaller


% Remove certain fields from bibliography
\DeclareSourcemap{
  \maps[datatype=bibtex, overwrite]{
    \map{  % Everywhere: Remove editors, [cs, eess] brackets, pubstate, DOI
      \step[fieldset=editor, null]  % Remove editors everywhere.
      \step[fieldset=pages, null]  % Remove pages everywhere.
      \step[fieldset=eprintclass, null]  % Remove [cs, eess] brackets on preprints.
      \step[fieldset=pubstate, null]  % Remove pubstate field on preprints
      \step[fieldset=doi, null]  % Remove pubstate field on preprints
      \step[fieldset=pmid, null]  % Remove pubstate field on preprints
    }
    \map{  % Remove URL on all types except online and document (which includes preprint and websites).
      \pernottype{online}
      \pernottype{document}
      \step[fieldset=url, null]
    }
    \map{  % Remove URL on entries that are preprints in effect (i.e. have an eprint field).
      \step[fieldsource=eprint, final]  % Applies to all entries with eprint field (terminates for those without it).
      \step[fieldset=url, null]  % Remove URL
    }
    \map{  % Remove conference title on conference papers
      \pertype{inproceedings}
      \step[fieldset=eventtitle, null]  % Remove conference title from conference papers (proceedings name is there)
    }
  }
}

\DefineBibliographyStrings{english}{%
  backrefpage = {cited on page},  % originally "cited on page"
  backrefpages = {cited on pages},  % originally "cited on pages"
}

% \newcommand{\printpublication}[1]{\AtNextCite{\defcounter{maxnames}{99}}\fullcite{#1}}  % create new command \printpublication that lists all authors
\preto\fullcite{\AtNextCite{\defcounter{maxnames}{99}}}  % make \fullcite list all authors

\DefineBibliographyExtras{english}{%
  \protected\def\mkbibdatelong#1#2#3{%
    \iffieldundef{#3}
      {}
      {\thefield{#3}%
       \iffieldundef{#2}{}{\nobreakspace}}%
    \iffieldundef{#2}
      {}
      {\mkbibmonth{\thefield{#2}}%
       \iffieldundef{#1}{}{\space}}%
    \iffieldbibstring{#1}{\bibstring{\thefield{#1}}}{\stripzeros{\thefield{#1}}}}%
}


% Highlight own author name
\usepackage{xstring}
\usepackage{etoolbox}
\newboolean{bold}
\newcommand{\makeauthorsbold}[1]{%
  \DeclareNameFormat{author}{%
  \setboolean{bold}{false}%
    \renewcommand{\do}[1]{\expandafter\ifstrequal\expandafter{\namepartfamily}{####1}{\setboolean{bold}{true}}{}}%
    \docsvlist{#1}%
    \ifthenelse{\value{listcount}=1}
    {%
      {\expandafter\ifthenelse{\boolean{bold}}{\mkbibbold{\namepartfamily\addcomma\addspace \namepartgiveni}}{\namepartfamily\addcomma\addspace \namepartgiveni}}%
    }{\ifnumless{\value{listcount}}{\value{liststop}}
      {\expandafter\ifthenelse{\boolean{bold}}{\mkbibbold{\addcomma\addspace \namepartfamily\addcomma\addspace \namepartgiveni}}{\addcomma\addspace \namepartfamily\addcomma\addspace \namepartgiveni}}%
      {\expandafter\ifthenelse{\boolean{bold}}{\mkbibbold{\addcomma\addspace \namepartfamily\addcomma\addspace \namepartgiveni\addcomma\isdot}}{\addcomma\addspace \namepartfamily\addcomma\addspace \namepartgiveni\addcomma\isdot}}%
      }
    \ifthenelse{\value{listcount}<\value{liststop}}
    {\addcomma\space}{}
  }
}
\makeauthorsbold{Havtorn}


% Plotting
\usepackage{pgfplots}
\pgfplotsset{compat=1.17}
\usepgfplotslibrary{fillbetween}
\pgfmathdeclarefunction{gauss}{2}
    {\pgfmathparse{1/(#2*sqrt(2*pi))*exp(-((x-#1)^2)/(2*#2^2))}}

% Colors
\usepackage{xcolor}
\definecolor{Blue}{rgb}{0.0,0.0,1.0}
\definecolor{Red}{rgb}{1.0,0.0,0.0}
\definecolor{MidnightBlue}{HTML}{006795}
\definecolor{RoyalBlue}{HTML}{0071BC}
\definecolor{OliveGreen}{HTML}{3C8031}

% Load preamble from thesis
\definecolor{customgreen} {RGB}{217	234	212}
\definecolor{customblue}  {RGB}{205	226	242}
\definecolor{custommorered}{RGB}{218 46 42}
\definecolor{customred}{RGB}{255 182 173}%{255 149 150}

\definecolor{presentation_green_shaded}{RGB}{217 234 211}
\definecolor{presentation_green}{RGB}{197 228 186}
\definecolor{presentation_blue_shaded}{RGB}{201 218 248}
\definecolor{presentation_blue}{RGB}{180 205 248}
\definecolor{presentation_orange}{RGB}{255 219 182}
\definecolor{presentation_orange_shaded}{RGB}{252 229 205}
\definecolor{presentation_red}{RGB}{255 182 173}
\definecolor{presentation_red_shaded}{RGB}{249 228 228}

\colorlet{observed-color}{customgreen}
\colorlet{latent-color}{customblue}
\colorlet{deterministic-color}{gray!15}
\colorlet{deterministic-skip-color}{custommorered}
\colorlet{shared-function-color}{blue}

\definecolor{theme-green}{RGB}{41 114 114}

% Checkmark and cross for brief overview paper
\usepackage{pifont}
\definecolor{ForestGreen}{RGB}{34,139,34}
\definecolor{xred}{HTML}{C91E12}
\definecolor{xorange}{RGB}{230,145,56}
\newcommand{\cmark}{{\color{ForestGreen}\ding{51}}}%
\newcommand{\xmark}{{\color{xred}\ding{55}}}%
\newcommand{\xmarkshaded}{{\color{xorange}\ding{55}}}%

% TikZ
\usepackage{tikz}
\usepackage[beamer,customcolors]{hf-tikz}
\usetikzlibrary{
    arrows,
    backgrounds,
    calc,
    matrix,
    positioning,
    shapes,
    shapes.multipart,
    topaths,
    bayesnet,
}
\tikzset{
state/.style={
       rectangle split,
       rectangle split parts=2,
       rectangle split part fill={red!30,blue!20},
       rounded corners,
       draw=black, very thick,
       minimum height=2em,
       text width=3cm,
       inner sep=2pt,
       text centered,
       }
}
\tikzset{hl/.style={
    set fill color=presentation_green_shaded,
    set border color=presentation_green_shaded,
  },
}


% Load preamble from thesis
%!TEX root = ../thesis.tex

% Text fonts (http://www.macfreek.nl/memory/Fonts_in_LaTeX)
% Install fonts from /usr/local/texlive/<version>/texmf-dist/fonts/opentype/public
\usepackage{fontspec}
\usepackage[T1]{fontenc}
\usepackage{lmodern}
\usepackage{slantsc}
% \RequirePackage{fix-cm}
\usepackage{bold-extra}
\usepackage{upgreek}

% \DeclareFontShape{T1}{texgyrepagella}{b}{sc}{<->ssub*cmr/bx/sc}{}
% \DeclareFontShape{T1}{texgyrepagella}{bx}{sc}{<->ssub*cmr/bx/sc}{}


% % Euler math fonts
% \usepackage[OT1,euler-digits,euler-hat-accent]{eulervm}
% \usepackage[bb=stixtwo]{mathalpha}
% \renewcommand{\mathbf}{\mathbold}  % euler requires \mathbold for bold math

% \usepackage{newtxmath}



% Serif font
% \usepackage{newpxtext}
% \setmainfont{TeX Gyre Pagella}
% \setmainfont{texgyrepagella}



\usepackage{mathpazo} % add possibly `sc` and `osf` options
\usepackage{eulervm}
\renewcommand{\mathbf}{\mathbold}  % euler requires \mathbold for bold math

% % Remove: "Font shape `T1/eulervm/m/n' undefined (Font) using `T1/cmr/m/n' instead."
% \usepackage{substitutefont}
% \substitutefont{TS1}{eulervm}{cmr}

%
% [
%   Extension=.otf,
%   UprightFont=*-regular,
%   ItalicFont=*-italic,
%   BoldFont=*-bold,
%   BoldItalicFont=*-bolditalic,
%   BoldSmallCapsFont=*-boldsmallcaps,
%   Numbers=OldStyle,
% ]

% \setmainfont{QTPalatine}

% \DeclareCharacterInheritance
%    { encoding = {TU,EU1,EU2},
%      family   = {QTPalatine} }
%    { A = {\`A,\'A,\^A,\~A,\"A,\r A},
%      a = {\`a,\'a,\^a,\~a,\"a,\r a},
%      C = {\c C},
%      c = {\c c},
%      D = {\DH},
%      d = {\dj},
%      E = {\`E,\'E,\^E,\"E},
%      e = {\`e,\'e,\^e,\"e},
%      I = {\`I,\'I,\^I,\"I},
%      i = {\`i,\'i,\^i,\"i,\i},
%      L = {\L},
%      l = {\l},
%      N = {\~N},
%      n = {\~n},
%      O = {\O,\`O,\'O,\^O,\~O,\"O},
%      o = {\o,\`o,\'o,\^o,\~o,\"o},
%      S = {\v S},
%      s = {\v s},
%      U = {\`U,\'U,\^U,\"U},
%      u = {\`u,\'u,\^u,\"u},
%      Y = {\'Y,\"Y},
%      y = {\'y,\"y},
%      Z = {\v Z},
%      z = {\v z}
%    }


% % Sans-serif font
% \setsansfont[
%     Ligatures=TeX,
%     Extension=.otf,
%     UprightFont=*-regular,
%     BoldFont=*-bold,
%     ItalicFont=*-italic,
%     BoldItalicFont=*-bolditalic,
%     % SlantedFont=,
%     % BoldSlantedFont=,
%     % SmallCapsFont=
%     Scale=0.8      % Adjustmens when using math in sections
% ]{texgyreadventor}


% Monospaced
% \setmonofont[Scale=MatchLowercase]{Linux Biolinum O}

%\setsansfont[Ligatures=TeX]{Neo Sans Intel}    % Neo Sans Intel – Like DTU font but more symbols
%\setsansfont[
%    Ligatures=TeX,
%    Scale=0.8
%]{NeoSans}           % NeoSans – DTU font (missing `+' symbols and other)
% \setsansfont[Ligatures=TeX]{CMU Sans Serif}    % Computer Modern Unicode font
%\setsansfont[Ligatures=TeX]{Latin Modern Sans} % Latin Modern Sans serif font

% Use this for more convienent sans serif font in math mode.
%\setmathsf{Latin Modern Sans}

% tikzlibrary.code.tex
%
% Copyright 2010-2011 by Laura Dietz
% Copyright 2012 by Jaakko Luttinen
%
% This file may be distributed and/or modified
%
% 1. under the LaTeX Project Public License and/or
% 2. under the GNU General Public License.
%
% See the files LICENSE_LPPL and LICENSE_GPL for more details.

% Load other libraries
\usetikzlibrary{shapes}
\usetikzlibrary{fit}
\usetikzlibrary{chains}
\usetikzlibrary{arrows}

% Latent node
\tikzstyle{latent} = [circle,fill=white,draw=black,inner sep=1pt,
minimum size=20pt, font=\fontsize{10}{10}\selectfont, node distance=1]
% Observed node
\tikzstyle{obs} = [latent,fill=gray!25]
% Constant node
\tikzstyle{const} = [rectangle, inner sep=0pt, node distance=1]
% Factor node
\tikzstyle{factor} = [rectangle, fill=black,minimum size=5pt, inner
sep=0pt, node distance=0.4]
% Deterministic node
\tikzstyle{det} = [latent, diamond]

% Plate node
\tikzstyle{plate} = [draw, rectangle, rounded corners, fit=#1]
% Invisible wrapper node
\tikzstyle{wrap} = [inner sep=0pt, fit=#1]
% Gate
\tikzstyle{gate} = [draw, rectangle, dashed, fit=#1]

% Caption node
\tikzstyle{caption} = [font=\footnotesize, node distance=0] %
\tikzstyle{plate caption} = [caption, node distance=0, inner sep=0pt,
below left=5pt and 0pt of #1.south east] %
\tikzstyle{factor caption} = [caption] %
\tikzstyle{every label} += [caption] %

\tikzset{>={triangle 45}}

%\pgfdeclarelayer{b}
%\pgfdeclarelayer{f}
%\pgfsetlayers{b,main,f}

% \factoredge [options] {inputs} {factors} {outputs}
\renewcommand{\factoredge}[4][]{ %
  % Connect all nodes #2 to all nodes #4 via all factors #3.
  \foreach \f in {#3} { %
    \foreach \x in {#2} { %
      \path (\x) edge[-,#1] (\f) ; %
      %\draw[-,#1] (\x) edge[-] (\f) ; %
    } ;
    \foreach \y in {#4} { %
      \path (\f) edge[->,#1] (\y) ; %
      %\draw[->,#1] (\f) -- (\y) ; %
    } ;
  } ;
}

% \edge [options] {inputs} {outputs}
\renewcommand{\edge}[3][]{ %
  % Connect all nodes #2 to all nodes #3.
  \foreach \x in {#2} { %
    \foreach \y in {#3} { %
      \path (\x) edge [->,#1] (\y) ;%
      %\draw[->,#1] (\x) -- (\y) ;%
    } ;
  } ;
}

% \factor [options] {name} {caption} {inputs} {outputs}
\renewcommand{\factor}[5][]{ %
  % Draw the factor node. Use alias to allow empty names.
  \node[factor, label={[name=#2-caption]#3}, name=#2, #1,
  alias=#2-alias] {} ; %
  % Connect all inputs to outputs via this factor
  \factoredge {#4} {#2-alias} {#5} ; %
}

% \plate [options] {name} {fitlist} {caption}
\renewcommand{\plate}[4][]{ %
  \node[wrap=#3] (#2-wrap) {}; %
  \node[plate caption=#2-wrap] (#2-caption) {#4}; %
  \node[plate=(#2-wrap)(#2-caption), #1] (#2) {}; %
}

% \gate [options] {name} {fitlist} {inputs}
\renewcommand{\gate}[4][]{ %
  \node[gate=#3, name=#2, #1, alias=#2-alias] {}; %
  \foreach \x in {#4} { %
    \draw [-*,thick] (\x) -- (#2-alias); %
  } ;%
}

% \vgate {name} {fitlist-left} {caption-left} {fitlist-right}
% {caption-right} {inputs}
\renewcommand{\vgate}[6]{ %
  % Wrap the left and right parts
  \node[wrap=#2] (#1-left) {}; %
  \node[wrap=#4] (#1-right) {}; %
  % Draw the gate
  \node[gate=(#1-left)(#1-right)] (#1) {}; %
  % Add captions
  \node[caption, below left=of #1.north ] (#1-left-caption)
  {#3}; %
  \node[caption, below right=of #1.north ] (#1-right-caption)
  {#5}; %
  % Draw middle separation
  \draw [-, dashed] (#1.north) -- (#1.south); %
  % Draw inputs
  \foreach \x in {#6} { %
    \draw [-*,thick] (\x) -- (#1); %
  } ;%
}

% \hgate {name} {fitlist-top} {caption-top} {fitlist-bottom}
% {caption-bottom} {inputs}
\renewcommand{\hgate}[6]{ %
  % Wrap the left and right parts
  \node[wrap=#2] (#1-top) {}; %
  \node[wrap=#4] (#1-bottom) {}; %
  % Draw the gate
  \node[gate=(#1-top)(#1-bottom)] (#1) {}; %
  % Add captions
  \node[caption, above right=of #1.west ] (#1-top-caption)
  {#3}; %
  \node[caption, below right=of #1.west ] (#1-bottom-caption)
  {#5}; %
  % Draw middle separation
  \draw [-, dashed] (#1.west) -- (#1.east); %
  % Draw inputs
  \foreach \x in {#6} { %
    \draw [-*,thick] (\x) -- (#1); %
  } ;%
}


\RequirePackage{xcolor}
% From https://www.dtu.dk/upload/dtu%20kommunikation/designguide/designplatform_farver2010.pdf
%
% Primally colors
\definecolor{dtured}{cmyk}{0,.91,.72,.23}
\definecolor{dtugray}{cmyk}{0,0,0,.56}
% Secondary colors
\definecolor{s12}{cmyk}{0,0.25,1,0}    % yellow
\definecolor{s01}{cmyk}{0,0.5,1,0}     % 
\definecolor{s02}{cmyk}{0,0.75,1,0}    % orange
\definecolor{s03}{cmyk}{0,1,1,0}       % red
\definecolor{s04}{cmyk}{0,1,1,0.5}     % 
\definecolor{s05}{cmyk}{.0,1,.0,.0}    % magenta
\definecolor{s06}{cmyk}{.25,1,.0,.0}   % 
\definecolor{s07}{cmyk}{.25,1,0,0}     % purple
\definecolor{s08}{cmyk}{.75,1,0,0}     % 
\definecolor{s09}{cmyk}{.75,.75,.0,.0} % 
\definecolor{s13}{cmyk}{.75,.50,.0,.0} % blue
\definecolor{s10}{cmyk}{.5,0,0,0}      % 
\definecolor{s11}{cmyk}{.5,.0,.0,.0}   % 
\definecolor{s14}{cmyk}{.5,0,1,0}      % green
% Tinted colors
\definecolor{s14a}{cmyk}{.6,0,1,.25}   % green

%!TEX root = ../thesis.tex

% This information is used in titlepage, colophon, preface and hyperref setup (pdf metainfo), and other options.

%\def\thesistypeabbr{B.Eng.}
%\def\thesistype    {Bachelor of Engineering}
%\def\thesistypeabbr{B.Sc.Eng.}
%\def\thesistype    {Bachelor of Science in Engineering}
%\def\thesistypeabbr{M.Sc.}
%\def\thesistype    {Master of Science in Engineering}
\def\thesistypeabbr{Ph.D.}
\def\thesistype    {Doctor of Philosophy}

\def\thesisdepshort{DTU Compute}
\def\thesisdep     {Department of Applied Mathematics and Computer Science}

\def\thesisauthor  {Jakob Drachmann Havtorn}
\def\thesistitle   {Uncertainty Estimation for Machine Learning Systems}
\def\thesissubtitle{Self-assessment on Medical Conversations}
\def\thesislocation{Copenhagen}
\def\thesisyear    {2023}
\def\thesismonth   {August}
\def\thesisday     {31}

\def\papersize     {b5paper} % Final papersize (b5paper/a4paper), recommended papersize for DTU Compute is b5paper
\def\showtrims     {false}   % Print on larger paper than \papersize and show trim marks (true/false)?
\def\showframe     {false}   % Show frame of writeable area, marginparsep and marginpar (true/false)?

\def\showtodos     {true}    % Show todos (true/false)?
\def\confidential  {false}   % Confidential thesis (true/false)?

\def\skippaperbody {true}    % Skip compiling paper bodies (true/false)?

%!TEX root = ../thesis.tex

% Parenthesis
\providecommand{\pa}[1]{{\left(#1\right)}}
  \renewcommand{\pa}[1]{{\left(#1\right)}}
\providecommand{\bra}[1]{{\left[#1\right]}}
  \renewcommand{\bra}[1]{{\left[#1\right]}}
\providecommand{\cbra}[1]{{\left\{#1\right\}}}
  \renewcommand{\cbra}[1]{{\left\{#1\right\}}}
\providecommand{\vbra}[1]{{\left\langle#1\right\rangle}}
  \renewcommand{\vbra}[1]{{\left\langle#1\right\rangle}}

% Matrices for displayed expressions
\providecommand{\mat}[1]{{\begin{matrix}#1\end{matrix}}}
  \renewcommand{\mat}[1]{{\begin{matrix}#1\end{matrix}}}
\providecommand{\pmat}[1]{{\begin{pmatrix}#1\end{pmatrix}}}
  \renewcommand{\pmat}[1]{{\begin{pmatrix}#1\end{pmatrix}}}
\providecommand{\bmat}[1]{{\begin{bmatrix}#1\end{bmatrix}}}
  \renewcommand{\bmat}[1]{{\begin{bmatrix}#1\end{bmatrix}}}

% Variations of \frac and \sfrac
\providecommand{\pfrac}[2]{{\left(\frac{#1}{#2}\right)}}
  \renewcommand{\pfrac}[2]{{\left(\frac{#1}{#2}\right)}}
\providecommand{\bfrac}[2]{{\left[\frac{#1}{#2}\right]}}
  \renewcommand{\bfrac}[2]{{\left[\frac{#1}{#2}\right]}}
\providecommand{\psfrac}[2]{{\left(\sfrac{#1}{#2}\right)}}
  \renewcommand{\psfrac}[2]{{\left(\sfrac{#1}{#2}\right)}}
\providecommand{\bsfrac}[2]{{\left[\sfrac{#1}{#2}\right]}}
  \renewcommand{\bsfrac}[2]{{\left[\sfrac{#1}{#2}\right]}}

% for small matrices to be used in in-line expressions
\providecommand{\sm}[1]{{\left\{#1\right\}}}
  \renewcommand{\sm}[1]{{\left\{#1\right\}}}
\providecommand{\psm}[1]{{\pa{\sm{#1}}}}
  \renewcommand{\psm}[1]{{\pa{\sm{#1}}}}
\providecommand{\bsm}[1]{{\bra{\sm{#1}}}}
  \renewcommand{\bsm}[1]{{\bra{\sm{#1}}}}

% Norm
\providecommand{\norm}[1]{{\left\lVert#1\right\rVert}}
  \renewcommand{\norm}[1]{{\left\lVert#1\right\rVert}}
% Size
\providecommand{\size}[1]{{\left\lvert#1\right\rvert}}
  \renewcommand{\size}[1]{{\left\lvert#1\right\rvert}}
% Trace
\providecommand{\Tr}[1]{{\text{Tr}\left[#1\right]}}
  \renewcommand{\Tr}[1]{{\text{Tr}\left[#1\right]}}
% Tranpose
\providecommand{\transpose}{{^\mathrm{T}}}
  \renewcommand{\transpose}{{^\mathrm{T}}}

% Derivatives
\providecommand{\od}[3][]{{\frac{\text{d}^{#1}#2}{\text{d}^{#1}#3}}}
  \renewcommand{\od}[3][]{{\frac{\text{d}^{#1}#2}{\text{d}^{#1}#3}}}
\providecommand{\pd}[3][]{{\frac{\partial^{#1}#2}{\partial^{#1}#3}}}
  \renewcommand{\pd}[3][]{{\frac{\partial^{#1}#2}{\partial^{#1}#3}}}

% Bold upright letters
\providecommand{\ab}{{\mathbf{a}}}
  \renewcommand{\ab}{{\mathbf{a}}}
\providecommand{\bb}{{\mathbf{b}}}
  \renewcommand{\bb}{{\mathbf{b}}}
\providecommand{\cb}{{\mathbf{c}}}
  \renewcommand{\cb}{{\mathbf{c}}}
\providecommand{\db}{{\mathbf{d}}}
  \renewcommand{\db}{{\mathbf{d}}}
\providecommand{\eb}{{\mathbf{e}}}
  \renewcommand{\eb}{{\mathbf{e}}}
\providecommand{\fb}{{\mathbf{f}}}
  \renewcommand{\fb}{{\mathbf{f}}}
\providecommand{\gb}{{\mathbf{g}}}
  \renewcommand{\gb}{{\mathbf{g}}}
\providecommand{\hb}{{\mathbf{h}}}
  \renewcommand{\hb}{{\mathbf{h}}}
\providecommand{\ib}{{\mathbf{i}}}
  \renewcommand{\ib}{{\mathbf{i}}}
\providecommand{\jb}{{\mathbf{j}}}
  \renewcommand{\jb}{{\mathbf{j}}}
\providecommand{\kb}{{\mathbf{k}}}
  \renewcommand{\kb}{{\mathbf{k}}}
\providecommand{\lb}{{\mathbf{l}}}
  \renewcommand{\lb}{{\mathbf{l}}}
\providecommand{\mb}{{\mathbf{m}}}
  \renewcommand{\mb}{{\mathbf{m}}}
\providecommand{\nb}{{\mathbf{n}}}
  \renewcommand{\nb}{{\mathbf{n}}}
\providecommand{\ob}{{\mathbf{o}}}
  \renewcommand{\ob}{{\mathbf{o}}}
\providecommand{\pb}{{\mathbf{p}}}
  \renewcommand{\pb}{{\mathbf{p}}}
\providecommand{\qb}{{\mathbf{q}}}
  \renewcommand{\qb}{{\mathbf{q}}}
\providecommand{\rb}{{\mathbf{r}}}
  \renewcommand{\rb}{{\mathbf{r}}}
\providecommand{\vs}{{\mathbf{s}}}
  \renewcommand{\vs}{{\mathbf{s}}}
\providecommand{\tb}{{\mathbf{t}}}
  \renewcommand{\tb}{{\mathbf{t}}}
\providecommand{\ub}{{\mathbf{u}}}
  \renewcommand{\ub}{{\mathbf{u}}}
\providecommand{\vb}{{\mathbf{v}}}
  \renewcommand{\vb}{{\mathbf{v}}}
\providecommand{\wb}{{\mathbf{w}}}
  \renewcommand{\wb}{{\mathbf{w}}}
\providecommand{\xb}{{\mathbf{x}}}
  \renewcommand{\xb}{{\mathbf{x}}}
\providecommand{\yb}{{\mathbf{y}}}
  \renewcommand{\yb}{{\mathbf{y}}}
\providecommand{\zb}{{\mathbf{z}}}
  \renewcommand{\zb}{{\mathbf{z}}}
\providecommand{\Ab}{{\mathbf{A}}}
  \renewcommand{\Ab}{{\mathbf{A}}}
\providecommand{\Bb}{{\mathbf{B}}}
  \renewcommand{\Bb}{{\mathbf{B}}}
\providecommand{\Cb}{{\mathbf{C}}}
  \renewcommand{\Cb}{{\mathbf{C}}}
\providecommand{\Db}{{\mathbf{D}}}
  \renewcommand{\Db}{{\mathbf{D}}}
\providecommand{\Eb}{{\mathbf{E}}}
  \renewcommand{\Eb}{{\mathbf{E}}}
\providecommand{\Fb}{{\mathbf{F}}}
  \renewcommand{\Fb}{{\mathbf{F}}}
\providecommand{\Gb}{{\mathbf{G}}}
  \renewcommand{\Gb}{{\mathbf{G}}}
\providecommand{\Hb}{{\mathbf{H}}}
  \renewcommand{\Hb}{{\mathbf{H}}}
\providecommand{\Ib}{{\mathbf{I}}}
  \renewcommand{\Ib}{{\mathbf{I}}}
\providecommand{\Jb}{{\mathbf{J}}}
  \renewcommand{\Jb}{{\mathbf{J}}}
\providecommand{\Kb}{{\mathbf{K}}}
  \renewcommand{\Kb}{{\mathbf{K}}}
\providecommand{\Lb}{{\mathbf{L}}}
  \renewcommand{\Lb}{{\mathbf{L}}}
\providecommand{\Mb}{{\mathbf{M}}}
  \renewcommand{\Mb}{{\mathbf{M}}}
\providecommand{\Nb}{{\mathbf{N}}}
  \renewcommand{\Nb}{{\mathbf{N}}}
\providecommand{\Ob}{{\mathbf{O}}}
  \renewcommand{\Ob}{{\mathbf{O}}}
\providecommand{\Pb}{{\mathbf{P}}}
  \renewcommand{\Pb}{{\mathbf{P}}}
\providecommand{\Qb}{{\mathbf{Q}}}
  \renewcommand{\Qb}{{\mathbf{Q}}}
\providecommand{\Rb}{{\mathbf{R}}}
  \renewcommand{\Rb}{{\mathbf{R}}}
\providecommand{\Sb}{{\mathbf{S}}}
  \renewcommand{\Sb}{{\mathbf{S}}}
\providecommand{\Tb}{{\mathbf{T}}}
  \renewcommand{\Tb}{{\mathbf{T}}}
\providecommand{\Ub}{{\mathbf{U}}}
  \renewcommand{\Ub}{{\mathbf{U}}}
\providecommand{\Vb}{{\mathbf{V}}}
  \renewcommand{\Vb}{{\mathbf{V}}}
\providecommand{\Wb}{{\mathbf{W}}}
  \renewcommand{\Wb}{{\mathbf{W}}}
\providecommand{\Xb}{{\mathbf{X}}}
  \renewcommand{\Xb}{{\mathbf{X}}}
\providecommand{\Yb}{{\mathbf{Y}}}
  \renewcommand{\Yb}{{\mathbf{Y}}}
\providecommand{\Zb}{{\mathbf{Z}}}
  \renewcommand{\Zb}{{\mathbf{Z}}}
% Bold upright numbers
\providecommand{\0}{{\mathbf{0}}}
  \renewcommand{\0}{{\mathbf{0}}}
\providecommand{\1}{{\mathbf{1}}}
  \renewcommand{\1}{{\mathbf{1}}}
\providecommand{\2}{{\mathbf{2}}}
  \renewcommand{\2}{{\mathbf{2}}}
\providecommand{\3}{{\mathbf{3}}}
  \renewcommand{\3}{{\mathbf{3}}}
\providecommand{\4}{{\mathbf{4}}}
  \renewcommand{\4}{{\mathbf{4}}}
\providecommand{\5}{{\mathbf{5}}}
  \renewcommand{\5}{{\mathbf{5}}}
\providecommand{\6}{{\mathbf{6}}}
  \renewcommand{\6}{{\mathbf{6}}}
\providecommand{\7}{{\mathbf{7}}}
  \renewcommand{\7}{{\mathbf{7}}}
\providecommand{\8}{{\mathbf{8}}}
  \renewcommand{\8}{{\mathbf{8}}}
\providecommand{\9}{{\mathbf{9}}}
  \renewcommand{\9}{{\mathbf{9}}}
% Bold upright greek symbols
\providecommand{\alphab}{{\boldsymbol{\upalpha}}}
  \renewcommand{\alphab}{{\boldsymbol{\upalpha}}}
\providecommand{\thetab}{{\boldsymbol{\uptheta}}}
  \renewcommand{\thetab}{{\boldsymbol{\uptheta}}}
\providecommand{\taub}{{\boldsymbol{\uptau}}}
  \renewcommand{\taub}{{\boldsymbol{\uptau}}}
\providecommand{\betab}{{\boldsymbol{\upbeta}}}
  \renewcommand{\betab}{{\boldsymbol{\upbeta}}}
\providecommand{\varthetab}{{\boldsymbol{\upvartheta}}}
  \renewcommand{\varthetab}{{\boldsymbol{\upvartheta}}}
\providecommand{\pib}{{\boldsymbol{\uppi}}}
  \renewcommand{\pib}{{\boldsymbol{\uppi}}}
\providecommand{\upsilonb}{{\boldsymbol{\upupsilon}}}
  \renewcommand{\upsilonb}{{\boldsymbol{\upupsilon}}}
\providecommand{\gammab}{{\boldsymbol{\upgamma}}}
  \renewcommand{\gammab}{{\boldsymbol{\upgamma}}}
\providecommand{\gammab}{{\boldsymbol{\upgamma}}}
  \renewcommand{\gammab}{{\boldsymbol{\upgamma}}}
\providecommand{\varpib}{{\boldsymbol{\upvarpi}}}
  \renewcommand{\varpib}{{\boldsymbol{\upvarpi}}}
\providecommand{\phib}{{\boldsymbol{\upphi}}}
  \renewcommand{\phib}{{\boldsymbol{\upphi}}}
\providecommand{\deltab}{{\boldsymbol{\updelta}}}
  \renewcommand{\deltab}{{\boldsymbol{\updelta}}}
\providecommand{\kappab}{{\boldsymbol{\upkappa}}}
  \renewcommand{\kappab}{{\boldsymbol{\upkappa}}}
\providecommand{\rhob}{{\boldsymbol{\uprho}}}
  \renewcommand{\rhob}{{\boldsymbol{\uprho}}}
\providecommand{\varphib}{{\boldsymbol{\upvarphi}}}
  \renewcommand{\varphib}{{\boldsymbol{\upvarphi}}}
\providecommand{\epsilonb}{{\boldsymbol{\upepsilon}}}
  \renewcommand{\epsilonb}{{\boldsymbol{\upepsilon}}}
\providecommand{\lambdab}{{\boldsymbol{\uplambda}}}
  \renewcommand{\lambdab}{{\boldsymbol{\uplambda}}}
\providecommand{\varrhob}{{\boldsymbol{\upvarrho}}}
  \renewcommand{\varrhob}{{\boldsymbol{\upvarrho}}}
\providecommand{\chib}{{\boldsymbol{\upchi}}}
  \renewcommand{\chib}{{\boldsymbol{\upchi}}}
\providecommand{\varepsilonb}{{\boldsymbol{\upvarepsilon}}}
  \renewcommand{\varepsilonb}{{\boldsymbol{\upvarepsilon}}}
\providecommand{\mub}{{\boldsymbol{\upmu}}}
  \renewcommand{\mub}{{\boldsymbol{\upmu}}}
\providecommand{\sigmab}{{\boldsymbol{\upsigma}}}
  \renewcommand{\sigmab}{{\boldsymbol{\upsigma}}}
\providecommand{\psib}{{\boldsymbol{\uppsi}}}
  \renewcommand{\psib}{{\boldsymbol{\uppsi}}}
\providecommand{\zetab}{{\boldsymbol{\upzeta}}}
  \renewcommand{\zetab}{{\boldsymbol{\upzeta}}}
\providecommand{\nub}{{\boldsymbol{\upnu}}}
  \renewcommand{\nub}{{\boldsymbol{\upnu}}}
\providecommand{\varsigmab}{{\boldsymbol{\upvarsigma}}}
  \renewcommand{\varsigmab}{{\boldsymbol{\upvarsigma}}}
\providecommand{\omegab}{{\boldsymbol{\upomega}}}
  \renewcommand{\omegab}{{\boldsymbol{\upomega}}}
\providecommand{\etab}{{\boldsymbol{\upeta}}}
  \renewcommand{\etab}{{\boldsymbol{\upeta}}}
\providecommand{\xib}{{\boldsymbol{\upxi}}}
  \renewcommand{\xib}{{\boldsymbol{\upxi}}}
\providecommand{\Gammab}{{\boldsymbol{\Upgamma}}}
  \renewcommand{\Gammab}{{\boldsymbol{\Upgamma}}}
\providecommand{\Lambdab}{{\boldsymbol{\Uplambda}}}
  \renewcommand{\Lambdab}{{\boldsymbol{\Uplambda}}}
\providecommand{\Sigmab}{{\boldsymbol{\Upsigma}}}
  \renewcommand{\Sigmab}{{\boldsymbol{\Upsigma}}}
\providecommand{\Psib}{{\boldsymbol{\Uppsi}}}
  \renewcommand{\Psib}{{\boldsymbol{\Uppsi}}}
\providecommand{\Deltab}{{\boldsymbol{\Updelta}}}
  \renewcommand{\Deltab}{{\boldsymbol{\Updelta}}}
\providecommand{\Xib}{{\boldsymbol{\Upxi}}}
  \renewcommand{\Xib}{{\boldsymbol{\Upxi}}}
\providecommand{\Upsilonb}{{\boldsymbol{\Upupsilon}}}
  \renewcommand{\Upsilonb}{{\boldsymbol{\Upupsilon}}}
\providecommand{\Omegab}{{\boldsymbol{\Upomega}}}
  \renewcommand{\Omegab}{{\boldsymbol{\Upomega}}}
\providecommand{\Thetab}{{\boldsymbol{\Uptheta}}}
  \renewcommand{\Thetab}{{\boldsymbol{\Uptheta}}}
\providecommand{\Pib}{{\boldsymbol{\Uppi}}}
  \renewcommand{\Pib}{{\boldsymbol{\Uppi}}}
\providecommand{\Phib}{{\boldsymbol{\Upphi}}}
  \renewcommand{\Phib}{{\boldsymbol{\Upphi}}}

% Bold letters
\providecommand{\abs}{{{\boldsymbol{a}}}}
  \renewcommand{\abs}{{{\boldsymbol{a}}}}
\providecommand{\bbs}{{{\boldsymbol{b}}}}
  \renewcommand{\bbs}{{{\boldsymbol{b}}}}
\providecommand{\cbs}{{{\boldsymbol{c}}}}
  \renewcommand{\cbs}{{{\boldsymbol{c}}}}
\providecommand{\dbs}{{{\boldsymbol{d}}}}
  \renewcommand{\dbs}{{{\boldsymbol{d}}}}
\providecommand{\ebs}{{{\boldsymbol{e}}}}
  \renewcommand{\ebs}{{{\boldsymbol{e}}}}
\providecommand{\fbs}{{{\boldsymbol{f}}}}
  \renewcommand{\fbs}{{{\boldsymbol{f}}}}
\providecommand{\gbs}{{{\boldsymbol{g}}}}
  \renewcommand{\gbs}{{{\boldsymbol{g}}}}
\providecommand{\hbs}{{{\boldsymbol{h}}}}
  \renewcommand{\hbs}{{{\boldsymbol{h}}}}
\providecommand{\ibs}{{{\boldsymbol{i}}}}
  \renewcommand{\ibs}{{{\boldsymbol{i}}}}
\providecommand{\jbs}{{{\boldsymbol{j}}}}
  \renewcommand{\jbs}{{{\boldsymbol{j}}}}
\providecommand{\kbs}{{{\boldsymbol{k}}}}
  \renewcommand{\kbs}{{{\boldsymbol{k}}}}
\providecommand{\lbs}{{{\boldsymbol{l}}}}
  \renewcommand{\lbs}{{{\boldsymbol{l}}}}
\providecommand{\mbs}{{{\boldsymbol{m}}}}
  \renewcommand{\mbs}{{{\boldsymbol{m}}}}
\providecommand{\nbs}{{{\boldsymbol{n}}}}
  \renewcommand{\nbs}{{{\boldsymbol{n}}}}
\providecommand{\obs}{{{\boldsymbol{o}}}}
  \renewcommand{\obs}{{{\boldsymbol{o}}}}
\providecommand{\pbs}{{{\boldsymbol{p}}}}
  \renewcommand{\pbs}{{{\boldsymbol{p}}}}
\providecommand{\qbs}{{{\boldsymbol{q}}}}
  \renewcommand{\qbs}{{{\boldsymbol{q}}}}
\providecommand{\rbs}{{{\boldsymbol{r}}}}
  \renewcommand{\rbs}{{{\boldsymbol{r}}}}
\providecommand{\sbs}{{{\boldsymbol{s}}}}
  \renewcommand{\sbs}{{{\boldsymbol{s}}}}
\providecommand{\tbs}{{{\boldsymbol{t}}}}
  \renewcommand{\tbs}{{{\boldsymbol{t}}}}
\providecommand{\ubs}{{{\boldsymbol{u}}}}
  \renewcommand{\ubs}{{{\boldsymbol{u}}}}
\providecommand{\vbs}{{{\boldsymbol{v}}}}
  \renewcommand{\vbs}{{{\boldsymbol{v}}}}
\providecommand{\wbs}{{{\boldsymbol{w}}}}
  \renewcommand{\wbs}{{{\boldsymbol{w}}}}
\providecommand{\xbs}{{{\boldsymbol{x}}}}
  \renewcommand{\xbs}{{{\boldsymbol{x}}}}
\providecommand{\ybs}{{{\boldsymbol{y}}}}
  \renewcommand{\ybs}{{{\boldsymbol{y}}}}
\providecommand{\zbs}{{{\boldsymbol{z}}}}
  \renewcommand{\zbs}{{{\boldsymbol{z}}}}
\providecommand{\Abs}{{{\boldsymbol{A}}}}
  \renewcommand{\Abs}{{{\boldsymbol{A}}}}
\providecommand{\Bbs}{{{\boldsymbol{B}}}}
  \renewcommand{\Bbs}{{{\boldsymbol{B}}}}
\providecommand{\Cbs}{{{\boldsymbol{C}}}}
  \renewcommand{\Cbs}{{{\boldsymbol{C}}}}
\providecommand{\Dbs}{{{\boldsymbol{D}}}}
  \renewcommand{\Dbs}{{{\boldsymbol{D}}}}
\providecommand{\Ebs}{{{\boldsymbol{E}}}}
  \renewcommand{\Ebs}{{{\boldsymbol{E}}}}
\providecommand{\Fbs}{{{\boldsymbol{F}}}}
  \renewcommand{\Fbs}{{{\boldsymbol{F}}}}
\providecommand{\Gbs}{{{\boldsymbol{G}}}}
  \renewcommand{\Gbs}{{{\boldsymbol{G}}}}
\providecommand{\Hbs}{{{\boldsymbol{H}}}}
  \renewcommand{\Hbs}{{{\boldsymbol{H}}}}
\providecommand{\Ibs}{{{\boldsymbol{I}}}}
  \renewcommand{\Ibs}{{{\boldsymbol{I}}}}
\providecommand{\Jbs}{{{\boldsymbol{J}}}}
  \renewcommand{\Jbs}{{{\boldsymbol{J}}}}
\providecommand{\Kbs}{{{\boldsymbol{K}}}}
  \renewcommand{\Kbs}{{{\boldsymbol{K}}}}
\providecommand{\Lbs}{{{\boldsymbol{L}}}}
  \renewcommand{\Lbs}{{{\boldsymbol{L}}}}
\providecommand{\Mbs}{{{\boldsymbol{M}}}}
  \renewcommand{\Mbs}{{{\boldsymbol{M}}}}
\providecommand{\Nbs}{{{\boldsymbol{N}}}}
  \renewcommand{\Nbs}{{{\boldsymbol{N}}}}
\providecommand{\Obs}{{{\boldsymbol{O}}}}
  \renewcommand{\Obs}{{{\boldsymbol{O}}}}
\providecommand{\Pbs}{{{\boldsymbol{P}}}}
  \renewcommand{\Pbs}{{{\boldsymbol{P}}}}
\providecommand{\Qbs}{{{\boldsymbol{Q}}}}
  \renewcommand{\Qbs}{{{\boldsymbol{Q}}}}
\providecommand{\Rbs}{{{\boldsymbol{R}}}}
  \renewcommand{\Rbs}{{{\boldsymbol{R}}}}
\providecommand{\Sbs}{{{\boldsymbol{S}}}}
  \renewcommand{\Sbs}{{{\boldsymbol{S}}}}
\providecommand{\Tbs}{{{\boldsymbol{T}}}}
  \renewcommand{\Tbs}{{{\boldsymbol{T}}}}
\providecommand{\Ubs}{{{\boldsymbol{U}}}}
  \renewcommand{\Ubs}{{{\boldsymbol{U}}}}
\providecommand{\Vbs}{{{\boldsymbol{V}}}}
  \renewcommand{\Vbs}{{{\boldsymbol{V}}}}
\providecommand{\Wbs}{{{\boldsymbol{W}}}}
  \renewcommand{\Wbs}{{{\boldsymbol{W}}}}
\providecommand{\Xbs}{{{\boldsymbol{X}}}}
  \renewcommand{\Xbs}{{{\boldsymbol{X}}}}
\providecommand{\Ybs}{{{\boldsymbol{Y}}}}
  \renewcommand{\Ybs}{{{\boldsymbol{Y}}}}
\providecommand{\Zbs}{{{\boldsymbol{Z}}}}
  \renewcommand{\Zbs}{{{\boldsymbol{Z}}}}
% Bold greek symbols
\providecommand{\alphabs}{{{\boldsymbol{\alpha}}}}
  \renewcommand{\alphabs}{{{\boldsymbol{\alpha}}}}
\providecommand{\thetabs}{{{\boldsymbol{\theta}}}}
  \renewcommand{\thetabs}{{{\boldsymbol{\theta}}}}
\providecommand{\taubs}{{{\boldsymbol{\tau}}}}
  \renewcommand{\taubs}{{{\boldsymbol{\tau}}}}
\providecommand{\betabs}{{{\boldsymbol{\beta}}}}
  \renewcommand{\betabs}{{{\boldsymbol{\beta}}}}
\providecommand{\varthetabs}{{{\boldsymbol{\vartheta}}}}
  \renewcommand{\varthetabs}{{{\boldsymbol{\vartheta}}}}
\providecommand{\pibs}{{{\boldsymbol{\pi}}}}
  \renewcommand{\pibs}{{{\boldsymbol{\pi}}}}
\providecommand{\upsilonbs}{{{\boldsymbol{\upsilon}}}}
  \renewcommand{\upsilonbs}{{{\boldsymbol{\upsilon}}}}
\providecommand{\gammabs}{{{\boldsymbol{\gamma}}}}
  \renewcommand{\gammabs}{{{\boldsymbol{\gamma}}}}
\providecommand{\gammabs}{{{\boldsymbol{\gamma}}}}
  \renewcommand{\gammabs}{{{\boldsymbol{\gamma}}}}
\providecommand{\varpibs}{{{\boldsymbol{\varpi}}}}
  \renewcommand{\varpibs}{{{\boldsymbol{\varpi}}}}
\providecommand{\phibs}{{{\boldsymbol{\phi}}}}
  \renewcommand{\phibs}{{{\boldsymbol{\phi}}}}
\providecommand{\deltabs}{{{\boldsymbol{\delta}}}}
  \renewcommand{\deltabs}{{{\boldsymbol{\delta}}}}
\providecommand{\kappabs}{{{\boldsymbol{\kappa}}}}
  \renewcommand{\kappabs}{{{\boldsymbol{\kappa}}}}
\providecommand{\rhobs}{{{\boldsymbol{\rho}}}}
  \renewcommand{\rhobs}{{{\boldsymbol{\rho}}}}
\providecommand{\varphibs}{{{\boldsymbol{\varphi}}}}
  \renewcommand{\varphibs}{{{\boldsymbol{\varphi}}}}
\providecommand{\epsilonbs}{{{\boldsymbol{\epsilon}}}}
  \renewcommand{\epsilonbs}{{{\boldsymbol{\epsilon}}}}
\providecommand{\lambdabs}{{{\boldsymbol{\lambda}}}}
  \renewcommand{\lambdabs}{{{\boldsymbol{\lambda}}}}
\providecommand{\varrhobs}{{{\boldsymbol{\varrho}}}}
  \renewcommand{\varrhobs}{{{\boldsymbol{\varrho}}}}
\providecommand{\chibs}{{{\boldsymbol{\chi}}}}
  \renewcommand{\chibs}{{{\boldsymbol{\chi}}}}
\providecommand{\varepsilonbs}{{{\boldsymbol{\varepsilon}}}}
  \renewcommand{\varepsilonbs}{{{\boldsymbol{\varepsilon}}}}
\providecommand{\mubs}{{{\boldsymbol{\mu}}}}
  \renewcommand{\mubs}{{{\boldsymbol{\mu}}}}
\providecommand{\sigmabs}{{{\boldsymbol{\sigma}}}}
  \renewcommand{\sigmabs}{{{\boldsymbol{\sigma}}}}
\providecommand{\psibs}{{{\boldsymbol{\psi}}}}
  \renewcommand{\psibs}{{{\boldsymbol{\psi}}}}
\providecommand{\zetabs}{{{\boldsymbol{\zeta}}}}
  \renewcommand{\zetabs}{{{\boldsymbol{\zeta}}}}
\providecommand{\nubs}{{{\boldsymbol{\nu}}}}
  \renewcommand{\nubs}{{{\boldsymbol{\nu}}}}
\providecommand{\varsigmabs}{{{\boldsymbol{\varsigma}}}}
  \renewcommand{\varsigmabs}{{{\boldsymbol{\varsigma}}}}
\providecommand{\omegabs}{{{\boldsymbol{\omega}}}}
  \renewcommand{\omegabs}{{{\boldsymbol{\omega}}}}
\providecommand{\etabs}{{{\boldsymbol{\eta}}}}
  \renewcommand{\etabs}{{{\boldsymbol{\eta}}}}
\providecommand{\xibs}{{{\boldsymbol{\xi}}}}
  \renewcommand{\xibs}{{{\boldsymbol{\xi}}}}
\providecommand{\Gammabs}{{{\boldsymbol{\Gamma}}}}
  \renewcommand{\Gammabs}{{{\boldsymbol{\Gamma}}}}
\providecommand{\Lambdabs}{{{\boldsymbol{\Lambda}}}}
  \renewcommand{\Lambdabs}{{{\boldsymbol{\Lambda}}}}
\providecommand{\Sigmabs}{{{\boldsymbol{\Sigma}}}}
  \renewcommand{\Sigmabs}{{{\boldsymbol{\Sigma}}}}
\providecommand{\Psibs}{{{\boldsymbol{\Psi}}}}
  \renewcommand{\Psibs}{{{\boldsymbol{\Psi}}}}
\providecommand{\Deltabs}{{{\boldsymbol{\Delta}}}}
  \renewcommand{\Deltabs}{{{\boldsymbol{\Delta}}}}
\providecommand{\Xibs}{{{\boldsymbol{\Xi}}}}
  \renewcommand{\Xibs}{{{\boldsymbol{\Xi}}}}
\providecommand{\Upsilonbs}{{{\boldsymbol{\Upsilon}}}}
  \renewcommand{\Upsilonbs}{{{\boldsymbol{\Upsilon}}}}
\providecommand{\Omegabs}{{{\boldsymbol{\Omega}}}}
  \renewcommand{\Omegabs}{{{\boldsymbol{\Omega}}}}
\providecommand{\Thetabs}{{{\boldsymbol{\Theta}}}}
  \renewcommand{\Thetabs}{{{\boldsymbol{\Theta}}}}
\providecommand{\Pibs}{{{\boldsymbol{\Pi}}}}
  \renewcommand{\Pibs}{{{\boldsymbol{\Pi}}}}
\providecommand{\Phibs}{{{\boldsymbol{\Phi}}}}
  \renewcommand{\Phibs}{{{\boldsymbol{\Phi}}}}

% \mathbb{} shortcuts
\providecommand{\abb}{{\mathbb{a}}}
  \renewcommand{\abb}{{\mathbb{a}}}
\providecommand{\bbb}{{\mathbb{b}}}
  \renewcommand{\bbb}{{\mathbb{b}}}
\providecommand{\cbb}{{\mathbb{c}}}
  \renewcommand{\cbb}{{\mathbb{c}}}
\providecommand{\dbb}{{\mathbb{d}}}
  \renewcommand{\dbb}{{\mathbb{d}}}
\providecommand{\ebb}{{\mathbb{e}}}
  \renewcommand{\ebb}{{\mathbb{e}}}
\providecommand{\fbb}{{\mathbb{f}}}
  \renewcommand{\fbb}{{\mathbb{f}}}
\providecommand{\gbb}{{\mathbb{g}}}
  \renewcommand{\gbb}{{\mathbb{g}}}
\providecommand{\hbb}{{\mathbb{h}}}
  \renewcommand{\hbb}{{\mathbb{h}}}
\providecommand{\ibb}{{\mathbb{i}}}
  \renewcommand{\ibb}{{\mathbb{i}}}
\providecommand{\jbb}{{\mathbb{j}}}
  \renewcommand{\jbb}{{\mathbb{j}}}
\providecommand{\kbb}{{\mathbb{k}}}
  \renewcommand{\kbb}{{\mathbb{k}}}
\providecommand{\lbb}{{\mathbb{l}}}
  \renewcommand{\lbb}{{\mathbb{l}}}
\providecommand{\mbb}{{\mathbb{m}}}
  \renewcommand{\mbb}{{\mathbb{m}}}
\providecommand{\nbb}{{\mathbb{n}}}
  \renewcommand{\nbb}{{\mathbb{n}}}
\providecommand{\obb}{{\mathbb{o}}}
  \renewcommand{\obb}{{\mathbb{o}}}
\providecommand{\pbb}{{\mathbb{p}}}
  \renewcommand{\pbb}{{\mathbb{p}}}
\providecommand{\qbb}{{\mathbb{q}}}
  \renewcommand{\qbb}{{\mathbb{q}}}
\providecommand{\rbb}{{\mathbb{r}}}
  \renewcommand{\rbb}{{\mathbb{r}}}
\providecommand{\sbb}{{\mathbb{s}}}
  \renewcommand{\sbb}{{\mathbb{s}}}
\providecommand{\tbb}{{\mathbb{t}}}
  \renewcommand{\tbb}{{\mathbb{t}}}
\providecommand{\ubb}{{\mathbb{u}}}
  \renewcommand{\ubb}{{\mathbb{u}}}
\providecommand{\vbb}{{\mathbb{v}}}
  \renewcommand{\vbb}{{\mathbb{v}}}
\providecommand{\wbb}{{\mathbb{w}}}
  \renewcommand{\wbb}{{\mathbb{w}}}
\providecommand{\xbb}{{\mathbb{x}}}
  \renewcommand{\xbb}{{\mathbb{x}}}
\providecommand{\ybb}{{\mathbb{y}}}
  \renewcommand{\ybb}{{\mathbb{y}}}
\providecommand{\zbb}{{\mathbb{z}}}
  \renewcommand{\zbb}{{\mathbb{z}}}
\providecommand{\Abb}{{\mathbb{A}}}
  \renewcommand{\Abb}{{\mathbb{A}}}
\providecommand{\Bbb}{{\mathbb{B}}}
  \renewcommand{\Bbb}{{\mathbb{B}}}
\providecommand{\Cbb}{{\mathbb{C}}}
  \renewcommand{\Cbb}{{\mathbb{C}}}
\providecommand{\Dbb}{{\mathbb{D}}}
  \renewcommand{\Dbb}{{\mathbb{D}}}
\providecommand{\Ebb}{{\mathbb{E}}}
  \renewcommand{\Ebb}{{\mathbb{E}}}
\providecommand{\Fbb}{{\mathbb{F}}}
  \renewcommand{\Fbb}{{\mathbb{F}}}
\providecommand{\Gbb}{{\mathbb{G}}}
  \renewcommand{\Gbb}{{\mathbb{G}}}
\providecommand{\Hbb}{{\mathbb{H}}}
  \renewcommand{\Hbb}{{\mathbb{H}}}
\providecommand{\Ibb}{{\mathbb{I}}}
  \renewcommand{\Ibb}{{\mathbb{I}}}
\providecommand{\Jbb}{{\mathbb{J}}}
  \renewcommand{\Jbb}{{\mathbb{J}}}
\providecommand{\Kbb}{{\mathbb{K}}}
  \renewcommand{\Kbb}{{\mathbb{K}}}
\providecommand{\Lbb}{{\mathbb{L}}}
  \renewcommand{\Lbb}{{\mathbb{L}}}
\providecommand{\Mbb}{{\mathbb{M}}}
  \renewcommand{\Mbb}{{\mathbb{M}}}
\providecommand{\Nbb}{{\mathbb{N}}}
  \renewcommand{\Nbb}{{\mathbb{N}}}
\providecommand{\Obb}{{\mathbb{O}}}
  \renewcommand{\Obb}{{\mathbb{O}}}
\providecommand{\Pbb}{{\mathbb{P}}}
  \renewcommand{\Pbb}{{\mathbb{P}}}
\providecommand{\Qbb}{{\mathbb{Q}}}
  \renewcommand{\Qbb}{{\mathbb{Q}}}
\providecommand{\Rbb}{{\mathbb{R}}}
  \renewcommand{\Rbb}{{\mathbb{R}}}
\providecommand{\Sbb}{{\mathbb{S}}}
  \renewcommand{\Sbb}{{\mathbb{S}}}
\providecommand{\Tbb}{{\mathbb{T}}}
  \renewcommand{\Tbb}{{\mathbb{T}}}
\providecommand{\Ubb}{{\mathbb{U}}}
  \renewcommand{\Ubb}{{\mathbb{U}}}
\providecommand{\Vbb}{{\mathbb{V}}}
  \renewcommand{\Vbb}{{\mathbb{V}}}
\providecommand{\Wbb}{{\mathbb{W}}}
  \renewcommand{\Wbb}{{\mathbb{W}}}
\providecommand{\Xbb}{{\mathbb{X}}}
  \renewcommand{\Xbb}{{\mathbb{X}}}
\providecommand{\Ybb}{{\mathbb{Y}}}
  \renewcommand{\Ybb}{{\mathbb{Y}}}
\providecommand{\Zbb}{{\mathbb{Z}}}
  \renewcommand{\Zbb}{{\mathbb{Z}}}

% \mathcal{} shortcuts
\providecommand{\ac}{{\mathcal{a}}}
  \renewcommand{\ac}{{\mathcal{a}}}
\providecommand{\bc}{{\mathcal{b}}}
  \renewcommand{\bc}{{\mathcal{b}}}
\providecommand{\cc}{{\mathcal{c}}}
  \renewcommand{\cc}{{\mathcal{c}}}
\providecommand{\dc}{{\mathcal{d}}}
  \renewcommand{\dc}{{\mathcal{d}}}
\providecommand{\ec}{{\mathcal{e}}}
  \renewcommand{\ec}{{\mathcal{e}}}
\providecommand{\fc}{{\mathcal{f}}}
  \renewcommand{\fc}{{\mathcal{f}}}
\providecommand{\gc}{{\mathcal{g}}}
  \renewcommand{\gc}{{\mathcal{g}}}
\providecommand{\hc}{{\mathcal{h}}}
  \renewcommand{\hc}{{\mathcal{h}}}
\providecommand{\ic}{{\mathcal{i}}}
  \renewcommand{\ic}{{\mathcal{i}}}
\providecommand{\jc}{{\mathcal{j}}}
  \renewcommand{\jc}{{\mathcal{j}}}
\providecommand{\kc}{{\mathcal{k}}}
  \renewcommand{\kc}{{\mathcal{k}}}
\providecommand{\lc}{{\mathcal{l}}}
  \renewcommand{\lc}{{\mathcal{l}}}
\providecommand{\mc}{{\mathcal{m}}}
  \renewcommand{\mc}{{\mathcal{m}}}
\providecommand{\nc}{{\mathcal{n}}}
  \renewcommand{\nc}{{\mathcal{n}}}
\providecommand{\oc}{{\mathcal{o}}}
  \renewcommand{\oc}{{\mathcal{o}}}
\providecommand{\pc}{{\mathcal{p}}}
  \renewcommand{\pc}{{\mathcal{p}}}
\providecommand{\qc}{{\mathcal{q}}}
  \renewcommand{\qc}{{\mathcal{q}}}
\providecommand{\rc}{{\mathcal{r}}}
  \renewcommand{\rc}{{\mathcal{r}}}
\providecommand{\sc}{{\mathcal{s}}}
  \renewcommand{\sc}{{\mathcal{s}}}
\providecommand{\tc}{{\mathcal{t}}}
  \renewcommand{\tc}{{\mathcal{t}}}
\providecommand{\uc}{{\mathcal{u}}}
  \renewcommand{\uc}{{\mathcal{u}}}
\providecommand{\vc}{{\mathcal{v}}}
  \renewcommand{\vc}{{\mathcal{v}}}
\providecommand{\wc}{{\mathcal{w}}}
  \renewcommand{\wc}{{\mathcal{w}}}
\providecommand{\xc}{{\mathcal{x}}}
  \renewcommand{\xc}{{\mathcal{x}}}
\providecommand{\yc}{{\mathcal{y}}}
  \renewcommand{\yc}{{\mathcal{y}}}
\providecommand{\zc}{{\mathcal{z}}}
  \renewcommand{\zc}{{\mathcal{z}}}
\providecommand{\Ac}{{\mathcal{A}}}
  \renewcommand{\Ac}{{\mathcal{A}}}
\providecommand{\Bc}{{\mathcal{B}}}
  \renewcommand{\Bc}{{\mathcal{B}}}
\providecommand{\Cc}{{\mathcal{C}}}
  \renewcommand{\Cc}{{\mathcal{C}}}
\providecommand{\Dc}{{\mathcal{D}}}
  \renewcommand{\Dc}{{\mathcal{D}}}
\providecommand{\Ec}{{\mathcal{E}}}
  \renewcommand{\Ec}{{\mathcal{E}}}
\providecommand{\Fc}{{\mathcal{F}}}
  \renewcommand{\Fc}{{\mathcal{F}}}
\providecommand{\Gc}{{\mathcal{G}}}
  \renewcommand{\Gc}{{\mathcal{G}}}
\providecommand{\Hc}{{\mathcal{H}}}
  \renewcommand{\Hc}{{\mathcal{H}}}
\providecommand{\Ic}{{\mathcal{I}}}
  \renewcommand{\Ic}{{\mathcal{I}}}
\providecommand{\Jc}{{\mathcal{J}}}
  \renewcommand{\Jc}{{\mathcal{J}}}
\providecommand{\Kc}{{\mathcal{K}}}
  \renewcommand{\Kc}{{\mathcal{K}}}
\providecommand{\Lc}{{\mathcal{L}}}
  \renewcommand{\Lc}{{\mathcal{L}}}
\providecommand{\Mc}{{\mathcal{M}}}
  \renewcommand{\Mc}{{\mathcal{M}}}
\providecommand{\Nc}{{\mathcal{N}}}
  \renewcommand{\Nc}{{\mathcal{N}}}
\providecommand{\Oc}{{\mathcal{O}}}
  \renewcommand{\Oc}{{\mathcal{O}}}
\providecommand{\Pc}{{\mathcal{P}}}
  \renewcommand{\Pc}{{\mathcal{P}}}
\providecommand{\Qc}{{\mathcal{Q}}}
  \renewcommand{\Qc}{{\mathcal{Q}}}
\providecommand{\Rc}{{\mathcal{R}}}
  \renewcommand{\Rc}{{\mathcal{R}}}
\providecommand{\Sc}{{\mathcal{S}}}
  \renewcommand{\Sc}{{\mathcal{S}}}
\providecommand{\Tc}{{\mathcal{T}}}
  \renewcommand{\Tc}{{\mathcal{T}}}
\providecommand{\Uc}{{\mathcal{U}}}
  \renewcommand{\Uc}{{\mathcal{U}}}
\providecommand{\Vc}{{\mathcal{V}}}
  \renewcommand{\Vc}{{\mathcal{V}}}
\providecommand{\Wc}{{\mathcal{W}}}
  \renewcommand{\Wc}{{\mathcal{W}}}
\providecommand{\Xc}{{\mathcal{X}}}
  \renewcommand{\Xc}{{\mathcal{X}}}
\providecommand{\Yc}{{\mathcal{Y}}}
  \renewcommand{\Yc}{{\mathcal{Y}}}
\providecommand{\Zc}{{\mathcal{Z}}}
  \renewcommand{\Zc}{{\mathcal{Z}}}



% Custom commands
\newcommand{\verticalmultirow}[2]{\parbox[t]{3mm}{\multirow{#1}{*}{\rotatebox[origin=c]{90}{#2}}}}
\newcommand{\tabitem}{{\color{dtured}$\bullet$} }
\newcommand{\highlight}[1]{{\color{dtured}#1}}

\newcommand\blfootnote[1]{%
    \begingroup
    \renewcommand\thefootnote{}\footnote{#1}%
    \addtocounter{footnote}{-1}%
    \endgroup
}

% The below is required to fix an issue where setting `\setbeameroption{show notes on second screen}` causes the content 
% frames (i.e. the ones with the actual content, not notes) that contain tables to render erroneously.
% Specifically, frames with tables that have more than one slide using <1->, <2->, etc. will only render table rules
% on the first slide and not on the following. Additionally, text in cells that have had their color set using e.g.
% `\color<1->{red}` also do not always render on slides other than the first. 
% Reference: https://github.com/josephwright/beamer/issues/337
\makeatletter
\def\beamer@framenotesbegin{% at beginning of slide
  \usebeamercolor[fg]{normal text}%
  \gdef\beamer@noteitems{}%
  \gdef\beamer@notes{}%
}
\makeatother
